\newpage

\section{Prerequisites}

\begin{itemize}

\item \textbf{Algorithm:} A step-by-step procedure for solving a certain \emph{problem}, e.g., ``sorting $n$ numbers'', ``find the closest coffee shop to your position''.

\item \textbf{Quality Criteria:} termination,  correctness, speed, memory, simplicity, generality, randomization, approximation Quality.

\item \textbf{Data Structure:} An organization of data such that certain operations can be handled efficiently. Ex: lists, heaps, etc.

\item \textbf{Quality Criteria:} memory, preprocessing time, query time, update time, simplicity

\end{itemize}

To evaluate the performance, we require a model.


\subsection{Random Access Model}

Memory consists of a sequence of cells, with an address ${addr} \in \mathbb{N}$. There are registers that can hold data, where operations may be performed. We assume the word size if polynomially bounded in $\log n$.

In this model, an algorithm is a numbered sequence of basic operations. The input of size $m$ for the algorithm is stored in memory $\{0, \ldots, m-1\}$. We have the following basic operations:

\begin{enumerate}
\item \textbf{load($i$, $j$):} Loads the content the memory cell with address stored in register $i$ into register $R_j$.
\item \textbf{store($i$, $j$):} Stores the content of register $i$ into the memory cell addressed by register $R_j$.
\item \textbf{assign($i$, $C$):} Stores $C$ into register $R_i$.
\item \textbf{increment/decrement($i$):} Increases/decreases value at $R_i$ by 1.
\item \textbf{op($i$, $j$, $k$):} Stores $R_i {op} R_j$ into $R_k$. ${op} \in \{+, \times, 0, \div, \mod, \wedge, \vee, \ldots\}$.
\item \textbf{jump($i$):} Jumps to operation $i$ in the algorithm.
\item \textbf{jump\_if($i$, $j$):} Jumps to operation $i$ in the algorithm if $R_j = 0$.
\end{enumerate}

\begin{algorithm}[h]
\begin{lstlisting}
assign(1, n) %*\quad*) -- 1
assign(2, 0) %*\quad*) -- 1
decrement(1) %*\quad*) -- n
load(1,3) %*\quad*) -- n
*(3,3,3) %*\quad*) -- n
+(3,2,2) %*\quad*) -- n
jump_if(8,1) %*\quad*) -- n
jump(2) %*\quad*) -- n-1
store(2,0) %*\quad*) -- 1
\end{lstlisting}
\caption{Add the squares of the first $n$ numbers in memory. The time complexity is $6n +2$.}
\end{algorithm}

\textbf{Cost Model}: Every basic operation takes one time unit to execute.

\begin{mydefinition}[Time Complexity]
The time complexity/runtime of algorithm $A$ for input $I$, $T_A(I)$, is the number of basic operations performed by $A$ on $I$.
\end{mydefinition}

With this definition of time complexity, we can compare speeds using natural functions.

\begin{mydefinition}[Worst-case]
The worst-case complexity $T_A^{wc}(n)$ of algorithm $A$ is equal to $\max\{I_A(I) | I \in I_n\}$.
\end{mydefinition}

\begin{mydefinition}[Best-case]
The best-case complexity $T_A^{wc}(n)$ of algorithm $A$ is equal to $\min\{I_A(I) | I \in I_n\}$.
\end{mydefinition}

\begin{mydefinition}[Average-case]
The average-case complexity $T_A^{wc}(n)$ of algorithm $A$ is equal to $\frac{1}{|I_n|} \sum\limits_{I \in I_N} T_A(I)$.
\end{mydefinition}

\subsection{O-notation}

\begin{mydefinition}
Let $f: \mathbb{N} \rightarrow \mathbb{R}_+$ be a function, then:

$$O(f(n)) = \{g: \mathbb{N}\rightarrow \mathbb{R}_+: \exists c>0, \exists n_0 \in \mathbb{N}, \forall n \ge n_0: g(n) \le c \cdot f(n)\}$$

$$\Omega(f(n)) = \{g: \mathbb{N}\rightarrow \mathbb{R}_+: \exists c>0, \exists n_0 \in \mathbb{N}, \forall n \ge n_0: g(n) \ge c \cdot f(n)\}$$

$$\Theta(f(n)) = O(f(n))\cap \Omega(f(n))$$

$$o(f(n)) = \{g: \mathbb{N}\rightarrow \mathbb{R}_+: \forall c>0, \exists n_0 \in \mathbb{N}, \forall n \ge n_0: g(n) \le c \cdot f(n)\}$$

$$\omega(f(n)) = \{g: \mathbb{N}\rightarrow \mathbb{R}_+: \forall c>0, \exists n_0 \in \mathbb{N}, \forall n \ge n_0: g(n) \ge c \cdot f(n)\}$$

\end{mydefinition}

Observation:
\begin{itemize}
\item $g \in O(f) \approx ``g \le f''$,
\item $g \in \Omega(f) \approx ``g \ge f''$,
\item $g \in \Theta(f) \approx ``g = f''$,
\item $g \in o(f) \approx ``g < f''$,
\item $g \in \omega(f) \approx ``g > f''$.
\end{itemize}

\begin{mylemma}
Let $p(n) = \sum\limits_{i=0}^k a_i \cdot n^i$ with $a_k > 0$. Then $p(n) \in \theta(n^k)$.
\end{mylemma}
\begin{proof}
\begin{enumerate}
\item (Show that $p \in O(n^k)$).

$$p(n) = \sum\limits_{i=0}^k a_i \cdot n^i \le \sum\limits_{i=0}^k |a_i| \cdot n^i
\le \sum\limits_{i=0}^k |a_i| \cdot n^k \le \left ( \sum\limits_{i=0}^k |a_i| \right )\cdot n^k \le c \cdot n^k$$

\item (Show that $p \in \Omega(n^k)$).

$$p(n) \ge a_k \cdot n^k - \sum\limits_{i=0}^{k-1} |a_i| \cdot n^i \ge
a_k \cdot n^k - n^{k-1} \cdot \sum\limits_{i=0}^{k-1} |a_i| =
\frac{a_k}{2} n^k +  \frac{a_k}{2} n^k - A \cdot n^{k-1} = \frac{a_k}{2} n^k + n^{k-1}(\frac{a_k}{2} n - A) \ge $$

\end{enumerate}
\end{proof}

\begin{mylemma}
If $L := \lim_{n \rightarrow \infty} \frac{f(n)}{g(n)}$ exists, then:

If $L = 0$ then $f(n) \in o(g(n))$. If $L \in (0, \infty)$ then $f(n) \in \Theta(g(n))$. If $L = \infty$ then $f(n) \in \omega(g(n))$.
\end{mylemma}
\begin{proof}
{\color{red} HOMEWORK!}
\end{proof}


\begin{mylemma}
$n \log(n) \in o(n^{1+\epsilon}$ for any $\epsilon > 0$.
\end{mylemma}
\begin{proof}
$\lim_{n \rightarrow \infty} \frac{n \log n}{n^{1+\epsilon}} =
\lim_{n \rightarrow \infty} \frac{\log n}{n^{\epsilon}} =
\lim_{n \rightarrow \infty} \frac{\frac{1}{n}}{\epsilon \cdot n^{\epsilon - 1}} =
\lim_{n \rightarrow \infty} \frac{1}{\epsilon \cdot n^{\epsilon}} = 0.$
\end{proof}

\begin{algorithm}
\begin{lstlisting}
Bubble_sort()
Input: a[1], ..., a[n]
Output: Sorted sequence
For i from 1 to n-1 do
    max_index = i
    For j from i+1 to n do
        if a[j] > a[max_index]
        then max_index = j
    end do
    if i != max_index then
        swap(a[i], a[max_index])
end do
\end{lstlisting}
\end{algorithm}

We compile this pseudocode to the machine model. This leads to the following cost model:

\begin{itemize}
\item cost(basic operation) = 1
\item cost(\textbf{if} A \textbf{then} B \textbf{else} C) = cost (A) + max\{cost(B), cost(C)\}
\item cost(\textbf{while} A \textbf{do} B) = (cost(A) + cost(B)) $\cdot$ iterations of the loop
\end{itemize}

Example. The cost of Bubble Sort is:

\begin{align*}
&\sum\limits_{i=0}^{n-1} (1 + (\sum\limits_{j=i+1}^n 3) + 3) = \sum\limits_{i=0}^{n-1} 4 + 3 \cdot \sum\limits_{j=i+1}^n 3 = \\
&= 4n + 3 \sum\limits_{i=0}^{n-1} (n-i) = 4n + 3 \sum\limits_{i=1}^{n} i = \ldots = \Theta(n^2) \\
\end{align*}
