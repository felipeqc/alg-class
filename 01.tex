\documentclass{article}
\usepackage{times}
\usepackage{amssymb}
\usepackage{amsmath}
\usepackage{amsthm}
\usepackage{amsxtra}
\usepackage{algorithm}
\usepackage{algpseudocode}
\usepackage{color}
\usepackage{listings}

\newtheorem{theorem}{Theorem}

\newtheorem{myexample}{\textbf{Example}}
\newtheorem{mylemma}{\textbf{Lemma}}
\newtheorem{myproof}{\textbf{Proof}}
\newtheorem{myinvariant}{\textbf{Invariant}}
\newtheorem{mytheorem}{\textbf{Theorem}}
\newtheorem{mycorollary}{\textbf{Corollary}}
\newtheorem{myapproach}{Approach}
\newtheorem{myproperty}{Property}
\newtheorem{mydefinition}{Definition}

\newtheorem{mycase}{Case}

\lstset{ %
  backgroundcolor=\color{white},   % choose the background color; you must add \usepackage{color} or \usepackage{xcolor}
  basicstyle=\footnotesize,        % the size of the fonts that are used for the code
  breakatwhitespace=false,         % sets if automatic breaks should only happen at whitespace
  breaklines=true,                 % sets automatic line breaking
  captionpos=b,                    % sets the caption-position to bottom
  commentstyle=\color{mygreen},    % comment style
  deletekeywords={...},            % if you want to delete keywords from the given language
  escapeinside={\%*}{*)},          % if you want to add LaTeX within your code
  extendedchars=true,              % lets you use non-ASCII characters; for 8-bits encodings only, does not work with UTF-8
  frame=single,                    % adds a frame around the code
  keepspaces=true,                 % keeps spaces in text, useful for keeping indentation of code (possibly needs columns=flexible)
  keywordstyle=\color{blue},       % keyword style
  language=Octave,                 % the language of the code
  morekeywords={assign, increment, decrement, jump, jump_if, store, *, +},            % if you want to add more keywords to the set
  numbers=left,                    % where to put the line-numbers; possible values are (none, left, right)
  numbersep=5pt,                   % how far the line-numbers are from the code
  rulecolor=\color{black},         % if not set, the frame-color may be changed on line-breaks within not-black text (e.g. comments (green here))
  showspaces=false,                % show spaces everywhere adding particular underscores; it overrides 'showstringspaces'
  showstringspaces=false,          % underline spaces within strings only
  showtabs=false,                  % show tabs within strings adding particular underscores
  stepnumber=1,                    % the step between two line-numbers. If it's 1, each line will be numbered
  stringstyle=\color{mymauve},     % string literal style
  tabsize=2,                       % sets default tabsize to 2 spaces
  title=\lstname                   % show the filename of files included with \lstinputlisting; also try caption instead of title
}


\title{Algorithms and Data Structures}

\begin{document}

\maketitle

\section{Stable Matching}

Discussed by Gale, Shapley in 1962 (context of job markets): This problem is also related the allocation of workers in hospitals.

Each company a strict preference over applicants. Applicant has a strict preference over companies.

Is there a solution that is self-enforcing?

\begin{mydefinition}[Self-enforcing]
A matching is self-enforcing if for every applicant A and company C, such that A is not assigned to C, either:
\begin{itemize}
\item C prefers every one of its accepted applicants over A, or 
\item A prefers his current situation over working for C.
\end{itemize}
\end{mydefinition}



\section{Formal Model}

\begin{itemize}
\item Set X of man, 	$n = |X|$

\item Set Y of woman,	$m = |Y|$

\item Each $x \in X$ has a strict preference order $>_x$ over $Y$.

\item Each $y \in Y$ has a strict preference order $>_y$ over $X$.

\item Matching $M \subseteq X \times Y$ is a set of ordered pairs such that
every agent appears in at most one pair.
\end{itemize}

A blocking pair $(x,y)$ is a pair such that either $x$ or $y$ prefers some other person than its current match. A stable matching does not have blocking pairs.

\textbf{Problem:} Does there always exist a stable matching? Can we compute it in polynomial time?

Note: There may be solutions that are unfair, that is, they satisfy only one side (companies or workers).

\section{Deferred Acceptance Algorithm}

\textbf{Main idea}: Starts with an empty set of edges.

\begin{algorithmic}[1]
    \State{Initially, all $x \in X$ and all $y \in Y$ are free.}
    \While{$\exists$ free man $x$ that has not proposed to all women}
        \State{Let $y$ be the highest-ranked proposed women for $x$ that he hasn't proposed to
}
        \State{$x$ proposes to $y$}
        \If{$y$ is free}
            \State{$(x,y)$ gets engaged}
        \Else
            \State{$y$ currently engaged to $x'$}
            \If{$x >_y x'$}
                \State{$(x,y)$ gets engaged and $x'$ becomes free}
            \EndIf
        \EndIf
    \EndWhile
\end{algorithmic}

\begin{theorem}
The DA algorithm terminates after $n \cdot m$ iterations.
\end{theorem}
\begin{proof}
Some straightforward properties:
\begin{enumerate}
\item If $y$ becomes engaged, she will be matched in the end. The engagement partners are improving with respect to her preference $>_y$ over time.
\item $x$ proposes in decreasing order of $>_x$.
Thus, number of iterations is upper bounded by $n \cdot m$.
\end{enumerate}

Let $y >_x y'$ and $x >_y x'$. By contradiction, suppose in the final matching there is a BP, that is, we have a matching $M=\{(x,y'), (x', y)\}$.

- Note that $y >_x y'$, so by algorithm, $x$ proposed to $y$.

- Then $x$ proposed to $y'$, but $x$ becomes free afterwards.

- $y$ had a better engaged partner $x' >_y x$.

Contradiction.
\end{proof}

\subsection{Formalisms}

\begin{enumerate}

\item Woman $y$ is a \emph{valid partner} for a man $x$ if there exists a SM $M$ with $(x,y) \in M$.

\item Valid partner $y$ is the \emph{best valid partner} for man $x$ if for every valid partner $y' \neq y$, we have that $y >_x y'$.

\item Similar for \emph{worst valid partner}.
\end{enumerate}

\textbf{Notation:} $y = best(x)$ and $best(x) = x$ if $x$ has no valid partner.

We define the optimal matching as follows:

$$M^+ = \{ (x, best(x) | x \in X \wedge best(x) \neq x\}$$

\begin{theorem}
Every execution of the DA algorithm results in $M^+$.
\end{theorem}
\begin{proof}
Contradiction. Assume there is an execution $E$ of the DA algorithm resulting in a SM $M$, where some man is not matched to his best valid partner. This means that this man was rejected by a better valid partner during $E$.

Consider the first time that this happens, i.e., a man $x$ is rejected by a valid partner.

We have $(x',y)$, with $x' >_y x$.

\begin{enumerate}
\item $x$ proposed in decreasing order of preference, $y$ is ${best}(x)$.
\item $y$ rejects $x$, so $y$ is engaged to $x'$, with $x' >_y x$.
\item There exists Sm M with $(x,y) \in M$. Let $(x', y') \in M$.
\item Since rejection by $y$ was first in $E$, then $x'$ is not rejected by any valid partner when engaged to $y$. Since $y'$ is a valid partner of $x'$, $y >_x y'$. Then $(x', y)$ is a BP for $M$. But we assumed that $M$ is stable! Contradiction.
\end{enumerate}
\end{proof}

\begin{theorem}
In $M^+$, every woman is matched to her worst valid partner (if any).
\end{theorem}
\begin{proof}
Suppose $(x,y) \in M^+$ such that $x$ is not the worst valid partner of $y$.
Then there exists an SM $M'$ with $(x', y) \in M'$ and $x >_y x'$.
In $M'$ we have $(x, y') \in M'$. $y = {best}(x)$ and $y'$ is a valid partner for $x$. So $y >_x y'$. Then $(x,y)$ is a BP in $M'$. Contradiction.
Similar if $x$ is unmatched in $M'$.
\end{proof}

\newpage

\section{Prerequisites}

\begin{itemize}

\item \textbf{Algorithm:} A step-by-step procedure for solving a certain \emph{problem}, e.g., ``sorting $n$ numbers'', ``find the closest coffee shop to your position''.

\item \textbf{Quality Criteria:} termination,  correctness, speed, memory, simplicity, generality, randomization, approximation Quality.

\item \textbf{Data Structure:} An organization of data such that certain operations can be handled efficiently. Ex: lists, heaps, etc.

\item \textbf{Quality Criteria:} memory, preprocessing time, query time, update time, simplicity

\end{itemize}

To evaluate the performance, we require a model.


\subsection{Random Access Model}

Memory consists of a sequence of cells, with an address ${addr} \in \mathbb{N}$. There are registers that can hold data, where operations may be performed. We assume the word size if polynomially bounded in $\log n$.

In this model, an algorithm is a numbered sequence of basic operations. The input of size $m$ for the algorithm is stored in memory $\{0, \ldots, m-1\}$. We have the following basic operations:

\begin{enumerate}
\item \textbf{load($i$, $j$):} Loads the content the memory cell with address stored in register $i$ into register $R_j$.
\item \textbf{store($i$, $j$):} Stores the content of register $i$ into the memory cell addressed by register $R_j$.
\item \textbf{assign($i$, $C$):} Stores $C$ into register $R_i$.
\item \textbf{increment/decrement($i$):} Increases/decreases value at $R_i$ by 1.
\item \textbf{op($i$, $j$, $k$):} Stores $R_i {op} R_j$ into $R_k$. ${op} \in \{+, \times, 0, \div, \mod, \wedge, \vee, \ldots\}$.
\item \textbf{jump($i$):} Jumps to operation $i$ in the algorithm.
\item \textbf{jump\_if($i$, $j$):} Jumps to operation $i$ in the algorithm if $R_j = 0$.
\end{enumerate}

\begin{algorithm}[h]
\begin{lstlisting}
assign(1, n) %*\quad*) -- 1
assign(2, 0) %*\quad*) -- 1
decrement(1) %*\quad*) -- n
load(1,3) %*\quad*) -- n
*(3,3,3) %*\quad*) -- n
+(3,2,2) %*\quad*) -- n
jump_if(8,1) %*\quad*) -- n
jump(2) %*\quad*) -- n-1
store(2,0) %*\quad*) -- 1
\end{lstlisting}
\caption{Add the squares of the first $n$ numbers in memory. The time complexity is $6n +2$.}
\end{algorithm}

\textbf{Cost Model}: Every basic operation takes one time unit to execute.

\begin{mydefinition}[Time Complexity]
The time complexity/runtime of algorithm $A$ for input $I$, $T_A(I)$, is the number of basic operations performed by $A$ on $I$.
\end{mydefinition}

With this definition of time complexity, we can compare speeds using natural functions.

\begin{mydefinition}[Worst-case]
The worst-case complexity $T_A^{wc}(n)$ of algorithm $A$ is equal to $\max\{I_A(I) | I \in I_n\}$.
\end{mydefinition}

\begin{mydefinition}[Best-case]
The best-case complexity $T_A^{wc}(n)$ of algorithm $A$ is equal to $\min\{I_A(I) | I \in I_n\}$.
\end{mydefinition}

\begin{mydefinition}[Average-case]
The average-case complexity $T_A^{wc}(n)$ of algorithm $A$ is equal to $\frac{1}{|I_n|} \sum\limits_{I \in I_N} T_A(I)$.
\end{mydefinition}

\subsection{O-notation}

\begin{mydefinition}
Let $f: \mathbb{N} \rightarrow \mathbb{R}_+$ be a function, then:

$$O(f(n)) = \{g: \mathbb{N}\rightarrow \mathbb{R}_+: \exists c>0, \exists n_0 \in \mathbb{N}, \forall n \ge n_0: g(n) \le c \cdot f(n)\}$$

$$\Omega(f(n)) = \{g: \mathbb{N}\rightarrow \mathbb{R}_+: \exists c>0, \exists n_0 \in \mathbb{N}, \forall n \ge n_0: g(n) \ge c \cdot f(n)\}$$

$$\Theta(f(n)) = O(f(n))\cap \Omega(f(n))$$

$$o(f(n)) = \{g: \mathbb{N}\rightarrow \mathbb{R}_+: \forall c>0, \exists n_0 \in \mathbb{N}, \forall n \ge n_0: g(n) \le c \cdot f(n)\}$$

$$\omega(f(n)) = \{g: \mathbb{N}\rightarrow \mathbb{R}_+: \forall c>0, \exists n_0 \in \mathbb{N}, \forall n \ge n_0: g(n) \ge c \cdot f(n)\}$$

\end{mydefinition}

Observation:
\begin{itemize}
\item $g \in O(f) \approx ``g \le f''$,
\item $g \in \Omega(f) \approx ``g \ge f''$,
\item $g \in \Theta(f) \approx ``g = f''$,
\item $g \in o(f) \approx ``g < f''$,
\item $g \in \omega(f) \approx ``g > f''$.
\end{itemize}

\begin{mylemma}
Let $p(n) = \sum\limits_{i=0}^k a_i \cdot n^i$ with $a_k > 0$. Then $p(n) \in \theta(n^k)$.
\end{mylemma}
\begin{proof}
\begin{enumerate}
\item (Show that $p \in O(n^k)$).

$$p(n) = \sum\limits_{i=0}^k a_i \cdot n^i \le \sum\limits_{i=0}^k |a_i| \cdot n^i
\le \sum\limits_{i=0}^k |a_i| \cdot n^k \le \left ( \sum\limits_{i=0}^k |a_i| \right )\cdot n^k \le c \cdot n^k$$

\item (Show that $p \in \Omega(n^k)$).

$$p(n) \ge a_k \cdot n^k - \sum\limits_{i=0}^{k-1} |a_i| \cdot n^i \ge
a_k \cdot n^k - n^{k-1} \cdot \sum\limits_{i=0}^{k-1} |a_i| =
\frac{a_k}{2} n^k +  \frac{a_k}{2} n^k - A \cdot n^{k-1} = \frac{a_k}{2} n^k + n^{k-1}(\frac{a_k}{2} n - A) \ge $$

\end{enumerate}
\end{proof}

\begin{mylemma}
If $L := \lim_{n \rightarrow \infty} \frac{f(n)}{g(n)}$ exists, then:

If $L = 0$ then $f(n) \in o(g(n))$. If $L \in (0, \infty)$ then $f(n) \in \Theta(g(n))$. If $L = \infty$ then $f(n) \in \omega(g(n))$.
\end{mylemma}
\begin{proof}
{\color{red} HOMEWORK!}
\end{proof}


\begin{mylemma}
$n \log(n) \in o(n^{1+\epsilon}$ for any $\epsilon > 0$.
\end{mylemma}
\begin{proof}
$\lim_{n \rightarrow \infty} \frac{n \log n}{n^{1+\epsilon}} =
\lim_{n \rightarrow \infty} \frac{\log n}{n^{\epsilon}} =
\lim_{n \rightarrow \infty} \frac{\frac{1}{n}}{\epsilon \cdot n^{\epsilon - 1}} =
\lim_{n \rightarrow \infty} \frac{1}{\epsilon \cdot n^{\epsilon}} = 0.$
\end{proof}

\begin{algorithm}
\begin{lstlisting}
Bubble_sort()
Input: a[1], ..., a[n]
Output: Sorted sequence
For i from 1 to n-1 do
    max_index = i
    For j from i+1 to n do
        if a[j] > a[max_index]
        then max_index = j
    end do
    if i != max_index then
        swap(a[i], a[max_index])
end do
\end{lstlisting}
\end{algorithm}

We compile this pseudocode to the machine model. This leads to the following cost model:

\begin{itemize}
\item cost(basic operation) = 1
\item cost(\textbf{if} A \textbf{then} B \textbf{else} C) = cost (A) + max\{cost(B), cost(C)\}
\item cost(\textbf{while} A \textbf{do} B) = (cost(A) + cost(B)) $\cdot$ iterations of the loop
\end{itemize}

Example. The cost of Bubble Sort is:

\begin{align*}
&\sum\limits_{i=0}^{n-1} (1 + (\sum\limits_{j=i+1}^n 3) + 3) = \sum\limits_{i=0}^{n-1} 4 + 3 \cdot \sum\limits_{j=i+1}^n 3 = \\
&= 4n + 3 \sum\limits_{i=0}^{n-1} (n-i) = 4n + 3 \sum\limits_{i=1}^{n} i = \ldots = \Theta(n^2) \\
\end{align*}


\section{Basic Data Structures}

\noindent \textbf{Goal:} Store ordered sequences $a_1, \ldots, a_k$ of elements.

\subsection{(Static) Array}

We can store elements in a contiguous region of memory. Element $a_i$ can be accessed in constant time.

\begin{itemize}
\item \textbf{(+)} Access to $i^{th}$ element
\item \textbf{(-)} Insert, delete
\end{itemize}

\subsection{Linked Lists}

\noindent Each \emph{item} contains:
\begin{enumerate}
\item An element $e$
\item Pointer to next item in the sequence
\item Pointer to the previous item in the sequence
\end{enumerate}

\noindent We also allocate a \emph{dummy item/head} that contains:
\begin{enumerate}
\item Stores $\bot$ (\emph{nil})
\item Next pointer points to first element
\item Prev pointer points to last elements
\end{enumerate}

A pointer to the \emph{head} item is used as a starting point for the operations (passed by parameter). We have the following operations:

\begin{itemize}
\item \emph{first\_element(h), last\_element(h), is\_empty(h)}: $\Theta(1)$
\item \emph{splice(a,b,t)}: Cuts entire sub-list $[a,b]$ and puts it between $t$ and $t'$. $\Theta(1)$
\item \emph{insert\_after(v, a)}: Takes element $v$ and inserts it after $a$. $\Theta(1)$
\item \emph{remove(a)}: Removes $a$ from its list. $\Theta(1)$
\item \emph{size of list}: It depends, since elements of sublists cannot be easily counted!
\item \emph{find(h, x)}: Is $x$ an element of the list? Returns pointer to the item that contains $x$, or $h$ if $x \notin $ list.
\end{itemize}

\begin{algorithm}
\begin{lstlisting}
insert_after(v,a):
    Create a list L with one item i, storing v.
    splice(v, v, a)
    Delete L.
\end{lstlisting}
\begin{lstlisting}
remove(a):
    Create empty list L with head h.
    splice(a, a, h)
    Delete L.
\end{lstlisting}
\begin{lstlisting}[mathescape]
find1(h,x):
    cur $\gets$ h->next
    while cur->e != x do
        if (run ==h) then break
        run $\gets$ cur->next
    end do
    return cur
\end{lstlisting}
\begin{lstlisting}[mathescape]
find2(h,x) [Sentinel Search]:
    (h->e) $\gets$ x
    cur $\gets$ h->next
    while cur->e != x do
        cur $\gets$ cur->next
    end do
    h->e $\gets \bot$
    return cur
\end{lstlisting}
\end{algorithm}

\newpage

\subsection{Unbounded Arrays}

We want to support
\begin{itemize}
\item Operator []: quick access to $i^{th}$ element
\item push\_back(v): append v to array
\item pop\_back(): Remove last element
\item size(): Returns number of elements
\end{itemize}

\noindent \textbf{Idea:} Static array of size $w$ for storing $n$ elements, with $w \ge n$. To fix the problem of having a limited array, we do a reallocation. We move all elements to a new array that is twice as large. Whenever $n \le \frac{w}{4}$ starts to be the common case, we reallocate the array to half size.

The problem is that push and pop can take:
\begin{itemize}
\item $\Theta(1)$ if no alloc is needed (good case)
\item $\Theta(n)$ if alloc is needed (bad case)
\end{itemize}

However, a bad case is preceded by roughly $n$ good cases to happen. In other words, the bad case is amortized by the good case. The following lemma shows that push-/pop-back operations have constant \emph{amortized} worst-case complexity.

\begin{mylemma}
A sequence of $m$ push-/pop-backs takes $\Theta(m)$ time.
\end{mylemma}
\begin{proof}[Proof (Bank Account Method)]
For any push-back we pay 2 tokens to a bank account. For any pop-back, we pay 1 token. We show that for a reallocation, we have enough tokens on the account to pay for moving the elements.

By induction. We initialize an empty array with $w=1$. $a_0$ adds 2. For the first reallocation, we have enough tokens to pay for the move.

After a reallocation, it holds that $n = w/2$. The next reallocation happens if (1) $n = w$ or if $n = w/4$. If $n=w$, we must have performed $w/2$ push-backs.
So, the bank account has at least $w$ tokens. enough for the $w$ movesin the next reallocation. If $n = w/4$, then we have $w/4$ pop-backs, each paying 1 token. So, the bank account has at least $w/4$ tokens. That is enough to pay for the move.
\end{proof}

\newpage

\subsection{Stacks, Queues, Deques}

\noindent Stacks support:
\begin{itemize}
\item push\_back(v)
\item pop\_back()
\item last()
\item size()
\end{itemize}

\noindent Queues support:
\begin{itemize}
\item push\_back(v)
\item pop\_front()
\item first()
\item size()
\end{itemize}

\noindent Queues support:
\begin{itemize}
\item push\_back(v)/push\_front(v)
\item pop\_front()/pop\_front()
\item first()/last()
\item size()
\end{itemize}

\noindent Lists (with restricted splice) can do all that in $\Theta(1)$.

\bigskip \noindent Stacks can be implemented with unbounded arrays.

\bigskip \noindent Queues can be implemented with circular arrays of fixed size $w$, using two pointers \emph{head} and \emph{tail}. Range $[h, t-1]$ stores the entries. Array is empty if $h = t$, and the size $n$ is given by $n = (t - h + w) \mod w$. We do reallocation as usual, by checking the size. This also works for deques.

\subsection{Basic Probability}

\textbf{Probability space $S$}: finite set with probabilities $P_s$ for $s \in S$, such that ${\sum_{s \in S} P_s = 1}$.

\noindent \textbf{Uniform probability space:} S is uniform if $\forall s \in S, P_s = 1/|S|$.

\noindent \textbf{Event:} $E \subseteq S$. ${prob}(E) \triangleq \sum_{S \in E} P_S$. If $S$ is uniform, ${prob}(E) = \frac{|E|}{|S|}$.

\noindent \textbf{Random Variables:} $X: S \rightarrow \mathbb{R}$
\begin{itemize}
\item can be composed: $X+Y, X\cdot Y, \ldots$
\item can be used for events: ${prob}(X \ge 5)$
\item indicator variables: $X: S \rightarrow \{0, 1\}$
\end{itemize}

\noindent \textbf{Expected Value:} $E[X] = \sum_{P_s X(s)} = \sum_{z \in \mathbb{R}} z \cdot {prob}(X = z)$.

\noindent \textbf{Linearity of expectation:} $E[X + Y] = E[X] + E[Y]$.

Variables $X_1,\ldots,X_k$ are independent if $\forall x_1,\ldots,x_k \in S$, ${prob}(X_1 = x_1 \wedge \cdots \wedge X_k = x_k) = \prod_{1 \le i \le k} {prob}(X_i = x_i)$.

\bigskip \noindent If $X$ and $Y$ are independent, then $E[X \cdot Y] = E[X] \cdot E[Y]$.

\bigskip \noindent \textbf{Markov Inequality:} For a non-negative $X$ and $x > 1$, ${prob}(X \ge k E[X]) \le 1/c$. 



\end{document}