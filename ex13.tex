\documentclass[a4paper]{article}
\usepackage[top=1in,bottom=1in,left=1in,right=1in]{geometry}
\usepackage{times}
\usepackage{amssymb}
\usepackage{mathtools}	% pulls in amsmath
	\mathtoolsset{centercolon}
\usepackage{tikz}
	\usetikzlibrary{automata}
	\usepackage{tikz-qtree}
\usepackage{mathpartir}
\usepackage{amsthm}
\usepackage{amsxtra}
\usepackage{algorithm}
\usepackage{algpseudocode}
\usepackage{semantic}
	\reservestyle{\declarevars}{\texttt}
	\reservestyle{\declareops}{\texttt}
	\reservestyle{\declarestates}{\text}
\usepackage{color}
\usepackage{listings}
\usepackage{mathtools}
\usepackage[shortlabels]{enumitem}
\usepackage{graphicx}	% for pdf image
\usepackage{pgfplots}
\pgfplotsset{width=7cm}

\newtheorem{theorem}{Theorem}

\newtheorem{myexample}{\textbf{Example}}
\newtheorem{mylemma}{\textbf{Lemma}}
\newtheorem{myproof}{\textbf{Proof}}
\newtheorem{myinvariant}{\textbf{Invariant}}
\newtheorem{mytheorem}{\textbf{Theorem}}
\newtheorem{mycorollary}{\textbf{Corollary}}
\newtheorem{myapproach}{Approach}
\newtheorem{myproperty}{Property}
\newtheorem{mydefinition}{Definition}

\newtheorem{mycase}{Case}

\lstset{ %
  backgroundcolor=\color{white},   % choose the background color; you must add \usepackage{color} or \usepackage{xcolor}
  basicstyle=\small,        % the size of the fonts that are used for the code
  breakatwhitespace=false,         % sets if automatic breaks should only happen at whitespace
  breaklines=true,                 % sets automatic line breaking
  captionpos=b,                    % sets the caption-position to bottom
  commentstyle=,    % comment style
  deletekeywords={...},            % if you want to delete keywords from the given language
  escapeinside={\%*}{*)},          % if you want to add LaTeX within your code
  extendedchars=true,              % lets you use non-ASCII characters; for 8-bits encodings only, does not work with UTF-8
 % frame=single,                    % adds a frame around the code
  keepspaces=true,                 % keeps spaces in text, useful for keeping indentation of code (possibly needs columns=flexible)
  columns=fullflexible,	% not monospace
  keywordstyle=,       % keyword style
  language=Octave,                 % the language of the code
  morekeywords={forall, to, else, then, end, and, or, assign, increment, decrement, jump, jump_if, store, *, +},            % if you want to add more keywords to the set
  numbers=left,                    % where to put the line-numbers; possible values are (none, left, right)
  numbersep=5pt,                   % how far the line-numbers are from the code
  rulecolor=\color{black},         % if not set, the frame-color may be changed on line-breaks within not-black text (e.g. comments (green here))
  showspaces=false,                % show spaces everywhere adding particular underscores; it overrides 'showstringspaces'
  showstringspaces=false,          % underline spaces within strings only
  showtabs=false,                  % show tabs within strings adding particular underscores
  stepnumber=1,                    % the step between two line-numbers. If it's 1, each line will be numbered
  stringstyle=,     % string literal style
  tabsize=4,                       % sets default tabsize to 2 spaces
  title=\lstname,                  % show the filename of files included with \lstinputlisting; also try caption instead of title
  mathescape,
  belowskip=-\baselineskip,
}

\DeclareMathOperator{\prob}{prob}
\DeclareMathOperator{\dom}{dom}
\DeclareMathOperator{\rank}{rank}
\DeclareMathOperator{\key}{key}
\newcommand*{\floor}[1]{\left\lfloor{#1}\right\rfloor}
\newcommand*{\ceil}[1]{\left\lceil{#1}\right\rceil}
\newcommand{\any}{{\rule[-.2ex]{1ex}{.4pt}}}	% Less hideous than \_.
\newcommand{\RR}{\mathbb{R}}
\newcommand{\NN}{\mathbb{N}}
\newcommand{\ZZ}{\mathbb{Z}}
\newcommand{\RP}{\RR_{\ge 0}}
\newcommand*{\dave}[1]{{\color{red}\textbf{PDS: #1}}}
\newcommand{\ie}{\emph{i.e.,} }
\newcommand{\eg}{\emph{e.g.,} }
\usepackage{hyperref}
\newcommand*{\Sref}[1]{\hyperref[#1]{\S\ref*{#1}}}
\newcommand*{\figref}[1]{\hyperref[#1]{Figure~\ref*{#1}}}
\newcommand{\edge}{\longrightarrow}
\newcommand{\redge}{\longleftarrow}

\title{Exercise Sheet 11---Algorithms and Data Structures}
\author{Felipe Cerqueira \\ 2547787 \and David Swasey \\ 2542105}

\begin{document}

\maketitle

Tutorial time: Monday 14:00


\section*{Exercise 1 (10 pts)}

The polar angle of a point $p_1$ with respect to $p_0$ is the angle that the vector $p_1 − p_0$ forms in the usual polar coordinate system. For instance, the polar angle of $(3, 5)$ with respect to $(2, 4)$ is the angle of $(1, 1)$, which is 45 degrees, or $\pi/4$. The polar angle of $(3, 3)$ with respect to $(2, 4)$ is the angle of $(1, −1)$, which is 315 degrees or $7\pi/4$. Design an $O(n \log n)$-algorithm that sorts a sequence of $n$ points $p_1, \ldots, p_n$ according to the polar angles with respect to a given origin $p_0$. Your algorithm should avoid computing the angles explicitly.

\section*{Exercise 2 (10 pts)}

A disk consists of a circle and its interior and is represented by its center point and (positive) radius. Two disks intersect if they have any point in common. Give an $O(n \log n)$-algorithm to determine whether any two disks in a set of $n$ disks intersect.

\section*{Exercise 3 (10 pts)}

For a set of $n$ segments, give an algorithm that lists all pairs of intersecting segments, sorted by the $x$-coordinate of the intersection point. You can suppose the same simplifying assumptions as in the lecture. With $k$ the number of intersection pairs, your algorithm should run in $O((k + n) \log n)$ time.

\section*{Exercise 4 (10 pts)}

Show by a counterexample that even if all consecutive triples in the (cyclic) sequence $(p_1, \ldots, p_n)$ form left turns, the induced polygon might not be convex. Give an $O(n)$-algorithm to decide whether $(p_1, \ldots, p_n)$ induces a convex polygon.

\section*{Exercise 5 (10 pts)}

In the online convex hull problem, we are given the set $Q$ of $n$ points one point at a time. After receiving each point, we have to compute the convex hull of the points seen so far. Obviously, we could re-run Graham’s scan every time, yielding a total running time of $O(n^2 \log n)$. Show how to solve the online convex hull problem in $O(n^2)$ time.

\paragraph{Answer.}

\end{document}
