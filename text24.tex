\section{Range Searching}

\paragraph{Problem:} Given a point set $Q$, design a data structure for the following queries:

\begin{itemize}
	\item For an arbitrary rectangle $R = [x,x']\times[y,y']$, return all points in $Q \cap R$.
\end{itemize}

\paragraph{Warm-up:} Given real numbers $\{x_1, \ldots, x_n\}$ and interval $I = [x, x']$, report all numbers in $I$.

Use Binary Search tree (BST):
\begin{itemize}
	\item Balanced (e.g., red-black-tree) $\rightarrow$ $O(\log n)$
	\item Points stored in leaves
	\item Leaves are \emph{not} connected by a list!
\end{itemize}

\begin{lstlisting}[mathescape]
${Find\_split\_node}(T, [x,x'])$:
	v \gets ${root}(T)$
	while $v$ is not a leaf and ($x' \le x_v$ or $x > x_v$):
		if $x' \le x_v$:
			$v \gets v_{left}$
		else:
			$v \gets v_{right}$
	return $v$

${Report\_subtree}(v)$: Report all leaves in the subtree rooted at $v$.

${1D\_range\_query}(T, [x,x']):$
	$v_{SPLIT} \gets {Find\_split\_node}(T, [x,x'])$
	if $v_{SPLIT}$ is a leaf, check whether point stored in leaf is in interval.
	$v \gets$ left child of $v_{SPLIT}$
	while $V$ is not a leaf:
		if $x \le x_v$:
			${Report\_subtree}(v_{right})$ (*)
			$v \gets v_{left}$
		else:
			$v \gets v_{right}$
	Check whether the point in leaf $v$ must be reported.
	Same for right child of $v_{SPLIT}$ and $x'$
\end{lstlisting}

\begin{mylemma}
A query runs in $O(k + \log n)$, where $k$ is the number of points to be reported.	
\end{mylemma}
\begin{proof}
Except for (*), a query runs in $O(\log n)$. Let $W$ be the set of nodes for which ${Report\_subtree}$ is called. The subtrees are disjoint. With $k_v$ as the nuber of leaves in a subtree, we have $\sum_{v \in W} k_v \le k$, so the running time is $O(k)$.
\end{proof}

\begin{itemize}
	\item {Preprocessing:} $O(n \log n)$
	\item {Storage: } $O(n)$
	\item {Query:} $O(k + \log n)$
\end{itemize}

\section{Range Trees}

What if we consider only $x$-coordinates?

\paragraph{Idea:} For points in $x$-range, we do a range query for $y$-coordinates.

\begin{mydefinition}
A \emph{range tree} $T$ for $Q$ is a BST storing the points in leaves and being sorted with respect to $x$-coordinate.
	
	For a node $v$ in $T$, we let $Q(v)$ denote the subset if points stored in the subtree rooted at $v$ (\emph{canonical subset} of $v$).
	
	Each internal node $v$ of $T$ stores a BST $T'(v)$ for $Q(v)$ storing the points in leaves sorted by $y$-coordinate (\emph{associated structure} of $v$).	
\end{mydefinition}

\begin{mylemma}
A range tree for $n$ points has size $O(n \log n)$.	
\end{mylemma}
\begin{proof}
	The size is the sum of the sizes of all associated structures. They are linear in the number of elements stored. A fixed point only belongs to $O(\log n)$ many associated structures.
\end{proof}

\paragraph{Construction:} Assume that points in $Q$ are sorted by $x$-coordinate.

\begin{lstlisting}[mathescape]
${Build\_Range\_Tree}(Q)$:
	If $|Q| = 1$, create a leaf node $v$ and store the point. Return $v$.
	Create node $v$ and set $T'(v)$ to a BST for $Q$ w.r.t. $y$-coordinate.
	Split $Q$ into $Q_{left}$, $Q_{right}$ of (roughly) same size
	$v_{left} \gets {Build\_Range\_Tree}(Q_{left})$ 
	$v_{right} \gets {Build\_Range\_Tree}(Q_{right})$
	return $v$	
\end{lstlisting}

\paragraph{Running Time:} $O(n \log ^2 n)$. But it can be improved to $O(n \log n)$ (skipped).

\paragraph{Query:} Same as 1D, but replace ${report\_subtree}$ by ${1D\_Range\_Query}$.

\begin{lstlisting}[mathescape]
${2D\_range\_query}(T, [x,x'] \times [y,y']):$
	$v_{SPLIT} \gets {Find\_split\_node}(T, [x,x'])$
	if $v_{SPLIT}$ is a leaf, check whether point stored in leaf is in interval.
	$v \gets$ left child of $v_{SPLIT}$
	while $V$ is not a leaf:
		if $x \le x_v$:
			${1D\_range\_query}(T'(v), [y,y'])$
			$v \gets v_{left}$
		else:
			$v \gets v_{right}$
	Check whether the point in leaf $v$ must be reported.
	Same for right child of $v_{SPLIT}$ and $x'$
\end{lstlisting}

\begin{mylemma}
${2D\_range\_query}$ runs in $O(k + \log^2 n)$.
\end{mylemma}
\begin{proof}
	$O(\log n)$ except of (*). Let $W$ be the set of nodes on which ${1D\_range\_query}$ is called. Let $k_v$ with $v \in W$ be the number of points reported from the canonical set $Q(v)$.
	
	Since $\sum_{v \in W} k_v \le k$, and one 1D-query takes $O(k_v + \log n)$ time, we have total cost $O(\sum_{v \in W} k_v + \log n) = O(k + \sum_{v \in W} \log n) = O(k + \log^2 n)$, because $|W| \le 2 \cdot {depth}(T) = O(\log n)$.
\end{proof}

\subsection{Fractional Cascading}

\paragraph{Goal:} Reduce query time to $O(k + \log n)$.

Given two sorted arrays, $A_1, A_2$ with $A_2 \subset A_1$, find all elements in $A_1$ and $A_2$that lie in some interval $I$.

We do binary search in $A_1$. But during preprocessing, we add points from $A_1$ to $A_2$ to the smallest element that is not smaller than the entry.

In range trees, $Q(v)$ is a superset of $Q(v_{left})$ and $Q(v_{right})$.

\paragraph{Layered Range Tree:} BST $T$ for $x$-coordinate, as before.

\paragraph{Associated Structure:} An array of points sorted by $y$-coordinate. Each entry has two pointers: one to the associated structure of the left child and one to the right child, pointing to the smallest element larger than or equal to the value of the entry.

\begin{lstlisting}[mathescape]
${Query}([x,x'] \times [y,y'])$:
	Find $v_{split}$ as before. Find entry $i$ in associated structure that contains the smallest value $\ge y$ (binary search).
	Follow search path for $x$ as before, and while descending follow pointers in the associated structure. When a 1D range query for $[y,y']$ is called, start reporting points beginning at the pointer until finding a value $> y$.
\end{lstlisting}

\paragraph{Running Time}: $O(1 + k_v)$ time per 1D-query on $[y,y']$ $\rightarrow$ $O(k + \log n)$ in total.

\subsection{Higher Dimensions}

For $n$ points in $d$ dimensions:
\begin{itemize}
	\item Construction: $O(n \log^{d-1} n)$
	\item Storage: $O(n \log^{d-1} n)$
	\item Query: $O(k + \log^d n)$ or $O(k + \log^{d-1} n)$ using fractional cascading
\end{itemize}

\paragraph{Alternative (kd-trees)}

One search tree, alternating splitting in $x$- and $y$-coordinate.

\begin{itemize}
	\item Construction: $O(n \log n)$ (2D), $O(n \log n)$ (general)
	\item Storage: $O(n)$ (2D), $O(n)$ (general)
	\item Query: $O(k + \sqrt{n})$ (2D), $O(k + n^{1 - \frac{1}{d}})$ (general)
\end{itemize}
