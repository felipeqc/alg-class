\section{Maximum Matching in General Graphs}

\paragraph{Bipartite Graphs:} Reduce to max-flow, but does not work for general graphs.

\subsection{Edmonds Algorithm}[Paths, Trees and Flowers (1965)]

Let $M$ be a matching.

\begin{itemize}
\item Alternating path: edges in and out of the matching
\item Augmenting path: Alternating path that starts and ends with free (unmatched) vertices, that can increase the matching size.
\end{itemize}

\begin{mylemma}
$M$ is a maximum matching $\iff$ $M$ admits no augmenting path.
\end{mylemma}
\begin{proof}
$M$ admits augmenting paths $\iff$ if $M$ is not maximum.

First, assume that $M$ admits augmenting path, then $M \oplus P$ increase matching size by 1.

Now, suppose $M$ is not maximum. Let $M^*$ be a maximum matching. Consider $M \oplus M^*$. In the symmetric difference, each vertex id incident to at most one edge from $M$ and at most one from $M^*$.

Each component of $M \oplus M*$ is one of the following: 
\begin{enumerate}
\item Isolated vertex. 
\item Path that starts with edge in $M$, alternates, and ends with an edge in $M^*$.
\item Path that starts with edge in $M^*$, alternates, and ends with an edge in $M$.
\item Cycle with even number of edges
\item Path that starts with edge in $M$, alternates, and ends with $M$.
\item Path that starts with edge in $M^*$, alternates, and ends with $M^*$.
\end{enumerate}

The first have the same number of edges from $M$ and $M^*$. (5) has more edges from $M$ than $M^*$. And (6) has more edges from $M^*$ than $M$.

The only component that satisfies $|M| < |M^*|$ is (6), which happens to be an augmenting path.
\end{proof}

\subsection{Maximum Matching Algorithm}

\begin{lstlisting}[mathescape]
    $M \gets \emptyset$
    while there is augmenting path $P$:
        $M \gets M \oplus P$
    return $M$
\end{lstlisting}

How to find an augmenting path?

\paragraph{Easy for bipartite graphs: } Start at an unmatched vertex $v$, then follow all incident unmatched edges. If the next vertex is free, we found an augmenting path. Otherwise, follow the matching edge to unique neighbor. Continue exploring unmatched edges.

Can be implemented using DFS, looking for alternating paths. But this doesn't work in general graphs, because there can be cycles with odd number of vertices.

DFS doesn't visit a node twice, which may prevent finding the augmenting path. But if we allow that edges are visited more than once, it also doesn't work.
Because it allows edges with repeated vertices (not a matching).

\paragraph{Edmunds idea:} Shrink cycles into super-vertices. Odd-cycle is called a blossom.

\begin{lstlisting}[mathescape]
    Start with free vertex. Label it 0.
    Search the graph for an alternating path:
        Labels alternate between 0 and 1
        If we are at 0-vertex:
            Search for unmatched edges to an already-seen 0-vertex.
            Shrink the blossom.
            Continue search.
        If we find unvisited tree-vertex, then augmenting paths exist.
    Unshrink blossoms in reversed order
    Construct augmenting path in odd cycles
\end{lstlisting}

\begin{itemize}
\item Unmatched 1-vertex: found augmenting path
\item From a 0-vertex. If we find a visited
    \begin{itemize}
    \item 0-vertex: found odd cycle, blossom!
    \item 1-vertex: found even cycle, ignore!
    \end{itemize}
\end{itemize}

\begin{mytheorem}
Let $H$ be obtained from $G$ by shrinking a blossom. Then, $H$ contains augmenting path $\iff$ $G$ contains augmenting path.
\end{mytheorem}
\begin{proof}
First, ``$\implies$''. In $H$, the path starts with 0-vertex and alternates until the base of the blossom. In $G$, the path starts with 0-vertex and alternates until we reach the odd cycle.

\begin{itemize}
\item If augmenting path in $H$ avoids base, then it exists in $G$ as well.
\item If augmenting path in $H$ goes through base, then we unshrink and go clockwise or counter-clockwise (only one is possible) to reach exist vertex of odd cycle in $G$. 
\end{itemize}

This shows how to construct path in $G$ if it exists in $H$.

Now, ``$\impliedby$''. If $G$ contains augmenting path, then $H$ contains augmenting path.

Consider $G$ and the augmenting path. We flip only the stem, which yields $G'$.
Note: $G$ has augmenting path $\implies$ $G'$ has augmenting path. Matching size is unchanged.

If $H'$ doesn't contain augmenting path $\iff$ $G'$ doesn't contain augmenting path.

\begin{enumerate}
\item If augmenting path avoids blossom, it also exists in $H'$
\item If augmenting path ends at base in $G'$, then it also ends in $H'$ as well (base is free)
\item If augmenting path contains part of blossom in $G'$, then truncating path at base gives augmenting path in $H'$.
\end{enumerate}
\end{proof}

More detailed description of the algorithm on the website

\begin{itemize}
\item Shrinking in blossom can be done in $O(m + n)$. The number of shrinks is $O(n)$.
\item Each augmenting path resolution increases matching size by 1.
\item Matching size $\le n/2$.
\item Loose bound: $n \cdot n \cdot (m + n) = O(m n^2) = O(n^4)$.
\item With fancy data structures: $O(n^3)$.
\item The best known algorithm is $O(m \cdot \sqrt{n})$ (Vazirani)
\end{itemize}

