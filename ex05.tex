\documentclass[a4paper]{article}
\usepackage[top=1in,bottom=1in,left=1in,right=1in]{geometry}
\usepackage{times}
\usepackage{amssymb}
\usepackage{mathtools}	% pulls in amsmath
	\mathtoolsset{centercolon}
\usepackage{tikz}
	\usetikzlibrary{automata}
	\usepackage{tikz-qtree}
\usepackage{mathpartir}
\usepackage{amsthm}
\usepackage{amsxtra}
\usepackage{algorithm}
\usepackage{algpseudocode}
\usepackage{semantic}
	\reservestyle{\declarevars}{\texttt}
	\reservestyle{\declareops}{\texttt}
	\reservestyle{\declarestates}{\text}
\usepackage{color}
\usepackage{listings}
\usepackage{mathtools}
\usepackage{enumerate}

\newtheorem{theorem}{Theorem}

\newtheorem{myexample}{\textbf{Example}}
\newtheorem{mylemma}{\textbf{Lemma}}
\newtheorem{myproof}{\textbf{Proof}}
\newtheorem{myinvariant}{\textbf{Invariant}}
\newtheorem{mytheorem}{\textbf{Theorem}}
\newtheorem{mycorollary}{\textbf{Corollary}}
\newtheorem{myapproach}{Approach}
\newtheorem{myproperty}{Property}
\newtheorem{mydefinition}{Definition}

\newtheorem{mycase}{Case}

\lstset{ %
  backgroundcolor=\color{white},   % choose the background color; you must add \usepackage{color} or \usepackage{xcolor}
  basicstyle=\small,        % the size of the fonts that are used for the code
  breakatwhitespace=false,         % sets if automatic breaks should only happen at whitespace
  breaklines=true,                 % sets automatic line breaking
  captionpos=b,                    % sets the caption-position to bottom
  commentstyle=\color{red},    % comment style
  deletekeywords={...},            % if you want to delete keywords from the given language
  escapeinside={\%*}{*)},          % if you want to add LaTeX within your code
  extendedchars=true,              % lets you use non-ASCII characters; for 8-bits encodings only, does not work with UTF-8
 % frame=single,                    % adds a frame around the code
  keepspaces=true,                 % keeps spaces in text, useful for keeping indentation of code (possibly needs columns=flexible)
  columns=fullflexible,	% not monospace
  keywordstyle=\color{blue},       % keyword style
  language=Octave,                 % the language of the code
  morekeywords={then, end, and, or, assign, increment, decrement, jump, jump_if, store, *, +},            % if you want to add more keywords to the set
  numbers=left,                    % where to put the line-numbers; possible values are (none, left, right)
  numbersep=5pt,                   % how far the line-numbers are from the code
  rulecolor=\color{black},         % if not set, the frame-color may be changed on line-breaks within not-black text (e.g. comments (green here))
  showspaces=false,                % show spaces everywhere adding particular underscores; it overrides 'showstringspaces'
  showstringspaces=false,          % underline spaces within strings only
  showtabs=false,                  % show tabs within strings adding particular underscores
  stepnumber=1,                    % the step between two line-numbers. If it's 1, each line will be numbered
  stringstyle=\color{mymauve},     % string literal style
  tabsize=2,                       % sets default tabsize to 2 spaces
  title=\lstname,                  % show the filename of files included with \lstinputlisting; also try caption instead of title
  mathescape
}

\DeclareMathOperator{\prob}{prob}
\DeclareMathOperator{\key}{key}
\newcommand*{\floor}[1]{\left\lfloor{#1}\right\rfloor}
\newcommand*{\ceil}[1]{\left\lceil{#1}\right\rceil}
\newcommand{\any}{{\rule[-.2ex]{1ex}{.4pt}}}	% Less hideous than \_.
\newcommand{\RR}{\mathbb{R}}
\newcommand{\NN}{\mathbb{N}}
\newcommand{\ZZ}{\mathbb{Z}}
\newcommand{\RP}{\RR_{\ge 0}}
\newcommand*{\dave}[1]{{\color{red}\textbf{PDS: #1}}}
\newcommand{\ie}{\emph{i.e.,} }
\newcommand{\eg}{\emph{e.g.,} }
\usepackage{hyperref}
\newcommand*{\figref}[1]{\hyperref[#1]{Figure~\ref*{#1}}}

\title{Exercise Sheet 3---Algorithms and Data Structures}
\author{Felipe Cerqueira \\ 2547787 \and David Swasey \\ 2542105}

\begin{document}

\maketitle

Tutorial time: Monday 14:00

\section{Exercise 1 (10 pts)}

The $i$-th Fibonacci number $F_i$ is defined by the recursion $F_0 = 0$, $F_1 = 1$, $F_i = F_{i-2} + F_{i-1}$.

\paragraph{a)} Show by induction that $F_{k+2} \ge \left ( \frac{1 + \sqrt{5}}{2} \right )^k$ for all $k \ge 0$.

\paragraph{Answer:}

By strong induction on $k$. For $k=0$, $F_{2} = F_0 + F_1 = 0 + 1 = 1$. Thus, $F_2 \ge \left ( \frac{1 + \sqrt{5}}{2} \right )^0 = 1$.

By IH, assume that for every $i \le k$, $F_{i+2} \ge \left ( \frac{1 + \sqrt{5}}{2} \right )^i$ holds. We need to prove it also holds for $k+1$.

Consider $F_{k+3} = F_{k+1} + F_{k+2}$. By IH, it follows that:

\begin{align*}
F_{(k+2) + 1} & = F_{k+1} + F_{k+2} && \text{[Substitution for $i=k-1$ and $i=k$]} \\
& \ge \left ( \frac{1 + \sqrt{5}}{2} \right )^{k-1} + \left ( \frac{1 + \sqrt{5}}{2} \right )^{k} \\
& = \left ( \frac{1 + \sqrt{5}}{2} \right )^{k-1} + \left ( \frac{1 + \sqrt{5}}{2} \right ) \cdot \left ( \frac{1 + \sqrt{5}}{2} \right )^{k-1} \\
& = \left (1 + \frac{1 + \sqrt{5}}{2} \right ) \cdot \left ( \frac{1 + \sqrt{5}}{2} \right )^{k-1} \\
& = \left ( \frac{3 + \sqrt{5}}{2} \right ) \cdot \left ( \frac{1 + \sqrt{5}}{2} \right )^{k-1} \\
& = \left ( \frac{1 + \sqrt{5}}{2} \right )^{2} \cdot \left ( \frac{1 + \sqrt{5}}{2} \right )^{k-1} \\
& = \left ( \frac{1 + \sqrt{5}}{2} \right )^{k+1}
\end{align*}


\paragraph{b)} Show that for all $k$:

$$F_{k+2} = 1 + \sum\limits_{i=0}^k F_i$$

\paragraph{Answer:}

For $k = 0$, $F_2 = 1 + \sum\limits_{i=0}^0 F_i = 1 + F_0 = 1$.

By IH, assume that for some $k \ge 0$, $F_{k+2} = 1 + \sum\limits_{i=0}^k F_i$. Let's prove this property holds for $k+1$.

\begin{align*}
1 + \sum\limits_{i=0}^{k+1} F_i & =  1 + \sum\limits_{i=0}^{k} F_i + F_{k+1} && \text{[By IH]} \\
& = F_{k+2} + F_{k+1} && \text{[Definition of Fibonacci sequence]} \\
& = F_{k+3}
\end{align*}

\paragraph{c)} Argue that for a sequence satisfying $S_0 = 1, S_1 \ge 2, S_k \ge 2 + S_0 + S_1 + \ldots + S_{k-2}$, it holds that $S_k \ge F_{k+2}$ for all $k \ge 0$.

\paragraph{Answer:}

By strong induction on $k$. For $k=0$, $S_0 = 1 \ge F_{2}$. For $k=1$, $S_1 \ge 2 \ge F_3$.

By IH, assume that for every $i \le k$, $S_i \ge F_{i+2}$. We must prove that this holds for $k+1$. By definition of $S$:

\begin{align*}
S_{k+1} & \ge 2 + S_0 + S_1 + \ldots + S_{k-1} && \text{[By IH]}\\
& \ge 2 + F_2 + F_3 + \ldots + F_{k+1} && \text{[$F_0 + F_1 = 1$]} \\
& \ge 1 + F_0 + F_1 + F_2 + F_3 + \ldots + F_{k+1} \\
& = 1 + \sum\limits_{i=0}^{k+1} F_i && \text{[By result from (b)]} \\
& = F_{k+3}
\end{align*}

\section{Exercise 3 (10 pts)}

Professor Pinocchio claims that the height of a Fibonacci heap with $n$ nodes is $O(\log n)$. Show that the professor is mistaken by giving, for any integer $n$, a sequence of addressable priority queue operations that creates a Fibonacci heap of just one tree that is a linear chain of $n$ nodes.

\paragraph{Answer:}

We can insert and extract a small dummy element (say, 0) to force roots to be combined. And then we use decrease-key/deletion to introduce some unbalance.

\begin{enumerate}
\item Insert 3, insert 4

\[
	\Tree [. \node{3};] \quad \Tree [. \node{4};]
\]

\item Insert 0, Extract-min 0

\[
	\Tree [.$3$ $4$ ]
\]

\item Insert 1, insert 2

\[
	\Tree [.$3$ $4$ ] \quad \Tree [. \node{1};] \quad \Tree [. \node{2};]
\]

\item Insert 0, Extract-min 0

\[
	\Tree [.$3$ $4$ ] \quad \Tree [.$1$ $2$ ]
\]

\newpage This is further combined:

\[
	\tikz{\Tree [.$1$ [.$3$ [.$4$ ] ] $2$ ]} \]

\item Delete 2

\[
	\tikz{\Tree [.$1$ [.$3$ [.$4$ ] ] ]} \]

\end{enumerate}

\section{Exercise 4 (10 pts)}

Show that with union by weights, but without path compression, a sequence of $n$ $make\_set$ and $m$ $union$ and $find$ operations takes $O(m \log n)$ time in the worst case. (Hint: prove the lemma from the lecture that the height of any tree is logarithmic in $n$).

\paragraph{Answer:}

Consider a sequence $\sigma$ of $m$ union/find operations on $n$ elements. For each operation $t$, let $T_t(u)$ denote the subtree rooted at $u$ after $t$ operations \emph{without} path compression. We define ${rank}(u) = 2 + h(T_m(u))$, where $h$ is the height.

In the class we proved that for every node $u$, $|T_t(u)| \ge 2^{h(T_t(u))}$, where $|T|$ denotes the number of nodes in $T$. By taking the logarithmic on both sides:

$$\log_2 (2^{h(T_t(u))}) \le \log_2 |T_t(u)|$$

Thus:

$$h(T_t(u)) \le \log_2 |T_t(u)| \le \log_2 n = O(\log n)$$

This shows that at any time, every subtree (including the root) has height $O(\log n)$.

\section{Exercise 5 (10 pts)}

The \emph{off-line minimum problem} asks us to maintain a set $T$ of elements from the domain $\{1, \ldots, n\}$ under the operations $insert$ and $extract\_min$, which returns and deletes the minimum element from the set. We are given a sequence $S$ of $n$ $insert$ and $m$ $extract\_min$ calls, where each key is inserted exactly once. We wish to fill an array $extracted[1,\ldots, m]$ of size $m$, where $extract[i]$ yields the value of the $i$-th call of $extract\_min$ call. The problem is ``off-line'', that is, we can process the entire sequence $S$ before determining any extracted key.

\paragraph{a)} An example sequence is given below. Here, an $insert$ is given a number and an $extract\_min$ is represented by the letter $E$:

$$4,8,E,3,E,9,2,6,E,E,E,1,7,E,5$$

What would be the correct $extract$ array to be returned?

\paragraph{Answer:} The array $[4, 3, 2, 6, 8, 1]$.

\paragraph{b)} Argue that the array returned by this method is correct.

\paragraph{Answer:} The method searches for the call where each element $i$ was inserted, and because this is done in increasing order, we guarantee that $i$ is the minimum element of the following ($j$-th) $extract\_min$ call. Since we know the result of this $j$-th call, it can be eliminated from the list, and the other inserted elements are left for being extracted by the remaining calls.

\paragraph{c)} Describe how to use a union-find data structure to implement the above algorithm efficiently and analyze the worst-case running time of your implementation. 

\paragraph{Answer:} First, we need to add an initialization phase where we read the array and extract the sets using $make\_set$ and $union$. We also maintain a doubly-linked of representative elements of each \textbf{non-empty} set $K_i$ ($i \le m)$: $<K_1 \rightarrow \cdots \rightarrow K_6  \rightarrow K_7>$, so that we can iterate over the sets more easily. Then we just update the algorithm:

\begin{itemize}
\item 3) Determine $j$ with $i \in K_j$:

This can be replaced by $j = find(i)$ to return the set where $i$ belongs to.

\item 4) if $j \neq m+1$:

This would check if the return value of the find(i) is valid (different from some dummy element).

\item 6 and 7) Set $K_l \gets K_\ell \cup K_j$ and remove the set $K_j$:

This can be replaced by locating $j$ in the linked-list (via some backward pointer to the handle, that we can store initially).

$<\cdots \rightarrow \underbrace{e(K_j)}_\text{here} \rightarrow e(K_\ell) \rightarrow \cdots>$.

Then, we do union of $K_j$ with the next set in the list and then delete $K_j$.

\end{itemize}

\paragraph{Complexity: } We need to add the initialization phase that reads the array and uses union/find/make-set operations ($O(m)$ amortized).
Then, inside the main loop ($1 \ldots n$), all the operations are pointer manipulation or union/find operations ($O(n)$ amortized).

The complexity of the algorithm is $O(m+n)$ amortized. 

\end{document}
