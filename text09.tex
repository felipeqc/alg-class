\section{Graph Algorithms}

\subsection{Connectivity}

We have a directed graph $G=(V,E)$, \emph{simple} (no loops or multi edges).

$$n = |V|, m = |E|$$

\paragraph{(Directed) Path of length $k$}

\paragraph{(Directed) Cycle of length $k+1$: } Directed path of length $k$ + find edge $(v_{k+1}, v_1)$.

\paragraph{Simple path, simple cycle: } No vertex appears twice on the path/cycle.

\begin{itemize}
\item Graph $G=(V, E)$ is \emph{strongly connected} if it has a $(v,u)$-path and $(u,v)$ path for all $v,w \in V$.

\item Undirected graph $G$ is $k$-connected ($k$-vertex connected) if removal of at most $k-1$ vertices (and all incident edges) keeps the graph connected. $k$-edge connected similar, but for edge removal.

\item An inclusion-maximal strongly connected subgraph is a strongly connected component. Same for $k$-connected and $k$-edge-connected. \emph{Intuitively, it cannot be extended with more vertices/edges and maintain the same property.}
\end{itemize}

\subsection{DFS Framework}

\begin{lstlisting}[mathescape]
Input: Graph directed or indirected, $G=(V,E)$
Data Structure:
    Stack S (for vertices on the DFS path)
    Vertex array "incoming" (for first incoming edge)
    Vertex & edge markings

for each $s \in V$ do:
    mark S, set incoming[s] $\gets$ nil, push $s \rightarrow S$
    root(S)
    
    while S is not empty do:
        $v \gets top(S)$
        if $\exists$ unmarked $e = (v,w) \in E$ then
            mark $e$
            if $w$ is unmarked:
                mark $w$, set incoming[$w$] $\gets$ e, push $w$
                traverse(v,e,w)
        else:
            $w \gets pop(S)$
            backtrack(w, incoming[w], top(S))
\end{lstlisting}

Sequence in which vertices are discovered/marked: $v_1, v_2, \ldots, v_n$.
\begin{itemize}
\item \emph{DFS-number} of $v_i$ as $DFS(v_i) = i$.
\item \emph{DFS-number} of $(v,w$) is $DFS((v,w)) = DFS(v)$.
\item \emph{DFS-order} $\preceq$ on $V$ or $E$ by:
\end{itemize}

$$DFS(p) \le DFS(q) \iff p \preceq q$$

$$DFS(p) < DFS(q) \iff p \prec q$$

\paragraph{Edge Classes: } If $(v,w)$ is marked, it becomes a
\begin{itemize}
\item \emph{tree edge} if $w$ is unmarked
\item \emph{back edge} if $w$ is marked, $w \preceq v, w \in S$
\item \emph{cross edge} if $w$ is marked, .......
\end{itemize}

\subsection{Undirected DFS}

\begin{itemize}
\item Edges are traversed independently of direction.
\item Replace the check of $\exists$ unmarked $e=(v,w) \in E$ with: $e=(v,w) \in E$ or $e=(w,v) \in E$.
\item In undirected DFS, there are no cross and no forward edges (homework!).
\end{itemize}

\subsection{Connected Components}

\begin{lstlisting}[mathescape]
root(s):
    $c \gets s$. component[S] $\gets c$

traverse(v,e,w):
    component[e] $\gets c$, component[w] $\gets c$.
\end{lstlisting}

\begin{itemize}
\item Component is represented by first discovered vertex.
\item Output: Array ``component'', pointing to representing vertex of the component
\item Requires $O(n+m)$ time, $O(n+m)$ space.
\end{itemize}

\subsection{2-connected Components}

\begin{mylemma}
For an undirected graph $h$, edges $e, e' \in E$ are in a 2-connected component if, and only if, there is a simple undirected cycle that contains both $e$ and $e'$. The set of 2-connected components is a partition of the $E$.
\end{mylemma}
\begin{proof}
If a cycle exists, then $e$ and $e'$ are in a 2-connected component.

Subdivide $e$ and $e'$ each with a new vertex. Component is still 2-connected.
Existence of a simple cycle follows from Menger's Theorem.
\end{proof}

\begin{mytheorem}[Menger's Theorem]
In a 2-connected undirected graph with at least 2 edges there are two vertex-disjoint paths between every pair of vertices.
\end{mytheorem}

\begin{proof}[Proof for partition statement]
For the partition statement, assume by contradiction $e$ is in two 2-connected components. There is a simple undirected cycle between $e'$ in one component and $e''$ in the other component. 
There are simple cycles, and they exist for every pair in the union of components. But components must be inclusion-maximal. Contradiction.
\end{proof}

\subsection{Undirected DFS for 2-connected components}

\begin{itemize}
\item Stack $S_E$ for edges in unfinished (or open) components
\item Stack $C$ for open components (represented by first edge)
\item Output: Array ``component'' pointing to representing edge
\end{itemize}

\begin{lstlisting}[mathescape]
traverse(v,e,w):
    push $e \rightarrow S_E$
    if $e$ is free edge then push $e \rightarrow C$
    if $e$ is back edge then:
        while $w \prec$ top(c) do:
            pop(c)

backtrack(w,e,v):
    if $e = top(c)$ and $e \neq nil$ then:
        pop(c)
        repeat
            $e' \gets pop(S_E)$
            component[$e'$] $\gets e$
        until $e = e'$.
\end{lstlisting}

\begin{mytheorem}
The DFS algorithm computes the 2-connected components of an undirected graph in time $O(n+m)$.
\end{mytheorem}
\begin{proof}
$Backtrack$ and $traverse$ add and remove each edge at most once for $C$ and $S_E$. Also, framework runs in time $O(n+m)$.

\textbf{Correctness:} By induction over the calls to $traverse$ and $backtrack$.

\begin{itemize}
\item Tree edge is called \emph{open} if there is no backtracking over it so far
\item or it is \emph{closed} otherwise.
\end{itemize}

Open edges induce a path. Every 2-connected-component has a unique first edge. Component is open/closed if first edge is open/closed, respectively.

After $t$ calls of $traverse$ and $backtrack$, we denote $G_t$ the subgraph induced by marked edges.

We denote $G_t^{(i)}$ for $i = 1, \ldots, k_t$. Edge set $E_t^{(i)}$ and first edge $e_t^{(i)}$.

\paragraph{Invariants}

\begin{enumerate}
\item Edges of closed components point (in component) to first edge.
\item On $C$ we have $e_t^{(i)} < \cdots < e_t^{(k)}$ in this order.
\item On $S_E$ we have edges from $E_t^{(1)}, \ldots, E_t^{(k_t)}$ in this order.
\end{enumerate}

All these invariants are true initially. Assume that these invariants hold until step $t-1$.

\paragraph{Step $t$ in two cases}

\begin{enumerate}
\item \textbf{Case 1 (call of $traverse(v,e,w)$)}: 
    \begin{enumerate}
    \item $e$ is tree edge: $w$ is vertex of degree 1 in $G_t$. $e$ is the only edge of a new open component. This implies that the invariants hold.
    \item $e$ is back edge: $e$ closes a simple cycle. The simple cycle can be found by traversing all tree edges on the DFS path up to $w$. All these edges belong to the same 2-connected component of $G_t$. All edges $e'$ with $w < e'$ are removed from $C$. Then, invariants hold.
    \end{enumerate}
    
\item \textbf{Case 2 (call of $backtrack(w,e,v)$)}:
    \begin{enumerate}
    \item $e = nil$: $w$ was put on $S$ by the outer loop of DFS. Then, all edges of the connected component and all 2-connected components are closed. $C$ and $S_E$ are empty.
    \item $e \neq top(c)$: Also ok.
    \item $e$ gets closed: By assumption, $e$ is the first edge of the open component with highest index $\implies$ on top of $C$. There are no cross or forward edges. Component cannot grow and its closed. Therefore, the repeat-loop deletes the correct edges from the stack and by assumption on the ordering of edges on $C$ and $S_E$, the invariants continue to hold.  
    \end{enumerate}
\end{enumerate}

\end{proof}

\subsection{2-edge Connected Components}

\begin{mylemma}
For an undirected graph $G$, two vertices $u,v \in V$ are in a 2-edge-connected component if and only if there is a closed sequence (cycle with repetition of vertices) of edges that contains both $u$ and $v$. The set of 2-edge-connected components is a partition of $V$. 
\end{mylemma}

\subsubsection{Undirected DFS for 2-edge connected components}

\begin{itemize}
\item $S_V$: vertices in open components
\item $C$: open component (represented by first vertex)
\item Output: Array component pointing to representing vertices
\end{itemize}

\begin{lstlisting}[mathescape]
root(s):
    push $s \rightarrow S_V$, push $s \rightarrow C$

traverse(v,e,w):
    if $e$ is tree edge then
        push $v \rightarrow S_V$, push $w \rightarrow C$
    if $e$ is back edge then:
        while $w < top(C)$ do:
            pop(C)

backtrack(w,e,v):
    if $w = top(C)$ then:
        pop(C)
        repeat
            $u \gets pop(S_V)$, $component[u] \gets w$
        until $u = w$
\end{lstlisting}

\begin{mytheorem}
The DFS algorithm computes the 2-edge-connected components of an undirected graph in time $O(n+m)$.
\end{mytheorem}
\begin{proof}
Exercise 1.
\end{proof}

\subsection{Strongly Connected Components}

\begin{mylemma}
For a directed graph $G$, two nodes $u,v \in V$ are in a SCC if and only if there is a closed directed sequence of edges (cycle with repetition of vertices) that contains both $u$ and $v$. The set of SCC forms a partition.
\end{mylemma}
\begin{proof}
Go for it.
\end{proof}

\subsubsection{Directed DFS for SCC}

Take DFS for 2-edge connected components and replace the ``if $e$ is back edge'' by:

``$e$ is back edge or $e$ is cross edge with $w \in S_V$''.

\begin{mytheorem}
The DFS algorithm computes the SCCs of a directed graph in time $O(n+m)$.
\end{mytheorem}
\begin{proof}
Exercise. Show that the algorithm correctly treats forward and cross edges.
\end{proof}

