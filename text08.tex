\newpage \section{Fibonacci Heaps}

\paragraph{Representation:}

\begin{itemize}
\item Forest of (arbitrary) heap-ordered trees
\item Min-pointer to smallest element ($\implies {min}(Q) = O(1)$)
\item Roots are connected by a linked list, called the \emph{root list}
\end{itemize}

\paragraph{Every node contains:}

\begin{itemize}
\item One element
\item Its rank (= \# children)
\item Pointers to parent, left and right sibling, and to one child
\item A flag (marked or not) used in ${decrease\_key}$.
\end{itemize}

\paragraph{Operations:}

\begin{itemize}
\item ${insert}(e,Q)$: Add new tree with one node $e$ to root list and update the ${min}$ pointer $\implies O(1)$.
\item ${union}(Q_1, Q_2):$ Splice the root lists and update the ${min}$ pointer $\implies O(1)$.
\item ${remove}(h, Q):$ ${decrease\_key}(h, -\infty, Q)$, then, ${delete\_min}(Q)$. 
\end{itemize}

\begin{lstlisting}[mathescape]
${delete\_min(Q)}$
    Delete min-node and add all its children to the root list
    Create and unbounded array $A$ storing handles to roots
    Traverse the root list. Let $v$ be a root and $r$ its rank:
        while($A[r] \neq \bot$):
            Let $w = A[v]$, so ${rank}(v)=r={rank}(w)$.
            Combine $v$ and $w$ into a new tree by making either $v$ a child
            of $w$ or vice versa.
            $A[v] \gets \bot$
            $r \gets r + 1$
        $A[v] \gets V$
    Traverse $A$ once more and find ${min-node}$ to update the pointer
            
\end{lstlisting}

${Decrease\_key}$ can be defined as follows:

\begin{lstlisting}[mathescape]
${decrease\_key(h, k, Q)}$
    Cut subtree at $h$ and make it a new root
    Update its key and update the ${min\_pointer}$. $\implies O(1)?$
\end{lstlisting}

However, this allows too ``wild'' shapes for the trees.



\begin{lstlisting}[mathescape]
Repair (Cascading Cuts):
    When node $v$ is cut, let $w$ be its parent.
    while($w$ is marked):
        Unmark and cut $w$
        $w \gets$ (old) parent of $w$
\end{lstlisting}

\subsection{Amortized Analysis}

\paragraph{Credit Invariant:} Each root holds one token and each marked node holds two tokens.

Let $R$ be the maximal rank of a node in the Fibonacci Heap.

\begin{mylemma}
${decrease\_key}$ needs to pay $O(1)$ credits to perform all its operations and maintain the credit invariant.
\end{mylemma}
\begin{proof}
The methods marks (at most) one element and we can spend the two tokens for this marking. The first cut costs $O(1)$. Any further cut turns a marked node into a root, so one token becomes free and we spend it to pay for the cut.
\end{proof}

\begin{mylemma}
${delete\_min}$ needs to pay $O(R)$ credits to pay for all its operations and to maintain the credit invariant.
\end{mylemma}
\begin{proof}
We need to pay $O(R)$ credits to make the children of min-node roots and give them one token each. The array $A$ has size $O(R)$, and ${delete\_min}$ can pay for initialization.

The running time of the loop is $O(R + \# {combines})$. Each combine turns some root into a non-root, so we can use its token to pay or the combine. Finding the min-pointer costs at most $O(R)$ additional operations.
\end{proof}

It remains to show that $R = O(\log n)$.

\paragraph{Fibonacci Numbers:}
\begin{itemize}
\item $F_0 = 0, F_1 = 1, \forall i \ge 2, FI = F_{i-1} + F_{i - 1}$
\item 0,1, 1,2,3,5,8,13,21, \ldots
\end{itemize}

\begin{mylemma}
$F_{i + 2} \ge \left ( \frac{1 + \sqrt{5}}{2} \right )^i \ge 1.618^i$ (homework).
\end{mylemma}

\begin{mylemma}
Let $v$ be a node of rank $i$. Then, the subtree rooted at $v$ has at least $F_{i+2}$ nodes. In particular, if the FH has $n$ nodes, each rank is bounded by $1.4404 \log n$.
\end{mylemma}
\begin{proof}
Let $v$ be a node of rank $i$. Let $w_1, \ldots, w_i$ be its children, sorted in order when they became children of $v$. For $w_j, 1 \le j \le i$, $w_j$ became child of $v$ through a combine in ${delete\_min}$. So, the rank of $w_j$ at that time was at least $j - 1$, because $w_1, \ldots, w_{j-1}$ were children of $v$ and combine only merges roots of same rank. Because $w_j$ has lost at most one child since then, ${rank}(w_j) \ge j-2$.

Let $S_i$ be the minimum number of nodes in a subtree with a root of rank $i$. We have $S_0 = 1, S_i \ge 2$ and $\forall i \ge 2, S_i \ge 2 + S_0 + S_1 + \ldots + S_{i-2}$.

We have that $S_i \ge F_{i+2}$ (homework).

For the second claim, for any ode of tank $i$, we have $F_{n+2} \le n \iff 1.618^n \le n \iff i \le 1.4404 \log n$.
\end{proof}

\section{Union-find}

We want to represent a collection of \emph{disjoint} sets $S_1, \ldots, S_k$. Each set $S_i$ has a representative $r_i \in S_i$. Our data structure supports:

\begin{itemize}
\item ${make\_set}(x):$ Create a new set $\{x\}$
\item ${union}(x, y):$ Combine $S_i$ and $S_j$, where $x \in S_i$ and $y \in S_j$, into a new set $S_i \cup S_j$ and pick some element of the union as representative.
\item ${find}(x):$ return the representative of the set containing $x$
\end{itemize}
 
\paragraph{Application:} Find the connected components of a graph $\implies$ minimum spanning tree.

\paragraph{Representation: }
\begin{itemize}
\item Forest of (arbitrary) trees, one per set
\item One element per node. Root contains the representative.
\item Only pointer to the parent. Root points to itself.
\end{itemize}

Naive implementation of ${make\_set}$, ${union}$, ${find}$ is clear. Efficiency? Bad, because trees can degenerate.

\paragraph{Optimization 1 (Union by weight): }
\begin{itemize}
\item For each root, maintain the number of nodes in its subtree (weight = 1 after ${make\_set}$, sum up the weights in ${union}$ operation).
\item When ${union}$ merges two trees with roots $x$, $y$, if $w(x) \le w(y)$, make $x$ a child of $y$, else make $y$ a child of $x$.
\end{itemize}

\paragraph{Optimization 2 (Path compression): } 
\begin{itemize}
\item In ${find}$, make all nodes in the search path children of the root.
\end{itemize}

\begin{mydefinition}[Ackermann Function]
For a function $f$ and $i \ge 0$, let $f^{(0)}(x) = x$ and $f^{(i+1)}(x) = f(f^{(i)}(x))$. Now, for $k\ge 0$ and $x \ge 2$,

\begin{align*}
A_0(x) = x+1 \\
A_{k+1}(x) = A_k^{(x)}(x)
\end{align*}

\end{mydefinition}

It follows that:

\begin{align*}
A_1(x) = 2x \\
A_2(x) = x \cdot 2^x \ge 2^x \\
A_3(x) = \underbrace{2^{2^2}}_{x} = 2 \uparrow x \\
A_3(x) = 2 \uparrow (2 \uparrow ( \ldots ) ) = 2 \uparrow \uparrow x
\end{align*}

Concretely, $A_0(2) = 3$, $A_1(2) = 4$, $A_2(2) = 8$, $A_3(2) = A_2(8) = 8 \cdot 2^8 = 2048$, $A_4(2) = A_3(A_3(2)) = A_3(2048) = \underbrace{2^{\cdots^2}}_2048$.

Define the inverse Ackermann function as:

\begin{align*}
\alpha(n) = \min \{ k \ge 0; A_k(2) \ge n \} \\
\alpha(n) \le 4, \forall n\le 2 \uparrow 2048
\end{align*}

\begin{mytheorem}
Using optimizations I and II, a sequence of $n$ make\_set and $m$ union-find operations takes $O((n+m)\alpha(n))$ time.
\end{mytheorem}

Consider a sequence $\sigma$ of $m$ union/find operations on $n$ elements. For $0 \le t \le m$, let $T_t(u)$ denote the subtree rooted at $u$ after $t$ operations \emph{without} path compression. Define:

\begin{align*}
{rank}(u) = 2 + h(T_m(u)) && \text{($h$ is the height)}
\end{align*}

If $u$ is a descendant of $v$ at some time $t$, then ${rank}(u) < {rank}(v)$.

\begin{mylemma}
For every node $u$, $|T_t(u)| \ge 2^{h(T_t(u))}$, where $|T|$ denotes the number of nodes in $T$.
\end{mylemma}
\begin{proof}
Induction on $t$. when $t=0$, for all $u$, $|T_t(u)| = 1 = 2^0 = 2^{h(T_t(u))}$.

If proposition is true for time $t$ and $h(T_{t+1}(u)) = h(T_t(u))$, then $|T_{t+1}(u)| \ge |T_t(u)| \underbrace{\ge}_\text{I.H.} 2^{h(T_t(u))} = 2^{h(T_{t+1}(u))}$.

\bigskip If $h(T_{t+1}(u)) > h(T_t(u))$, the $(t+1)-th$ operation is a union that makes another root $v$ a child of $u$. 

Because of union by weights,  $|T_t(v)| \le |T_t(u)|$ and also $h(T_t(v)) = h(T_{t+1}(v)) = h(T_{t+1}(u))-1$.

So, $|T_{t+1}(u)| = |T_t(u)| + |T_t(v)| \ge 2 |T_t(v)| \ge 2 \cdot 2^{h(T_t(v))} = 2^{h(T_{t+1}(u))}$ 
\end{proof}

\begin{mycorollary}
For all $u$, ${rank}(u) \le \lfloor \log n \rfloor + 2$. (HOMEWORK)
\end{mycorollary}

\begin{mylemma}
$\sum\limits_{u} {rank}(u) \le 8n$
\end{mylemma}
\begin{proof}
Let $c+r$ be the number of nodes of rank $r$. Any pair of such nodes has disjoint subtrees at time $m$. Since each subtree has at least $2^{t-2}$ nodes, $c_r \le \frac{n}{2^{r-2}}$.

$$\sum\limits_{u} {rank}(u) \le \sum\limits_{i=0}^\infty i \cdot \underbrace{|\{u | {rank}(u) = i\}|}_{c_i} \le \sum i \cdot \frac{n}{2^{i-2}} \le 4n \sum \frac{i}{2^i} \le 8n$$

\end{proof}

Now, consider the same sequence $\sigma$ \emph{with path compression}. Define

\begin{align*}
\delta(u) = \min \{k : {rank}({par}(u)) \ge A_k({rank}(u)) \} && \text{[par is the parent]}
\end{align*}

Note that $\delta(u)$ depends on time $t$ (because parent changes due to path compression). $\delta(u)$ cannot decrease over time.

\begin{mylemma}
For $n \ge 5$, $\delta(u) < \alpha(n)$
\end{mylemma}
\begin{proof}
For $n \ge 5$:

\begin{align*}
n &> \lfloor \log n \rfloor + 2 \\
& \ge {rank}({par}(u)) \\
& \ge A_{\delta(u)}({rank}(u)) \\
& \ge A_{\delta(u)}(2)
\end{align*}

But $A_{\alpha(n)} (2) \ge n > A_{\delta(u)}(2)$, so $\alpha(n) > \delta(u)$.
\end{proof}

We give each node $u$, $\alpha(n) \cdot {rank}(u)$ tokens initially. This makes $\sum_u \alpha(n) \cdot {rank}(u) \le 8 \cdot \alpha(n) \cdot n = O(n \cdot \alpha(n))$.

For a node $x$ on the search path of a find,
\begin{enumerate}
\item if $x$ has an ancestor $y$ with $\delta(x) = \delta(y)$, we spend one token of $x$ to pay for traversing $x$ and its path compression.
\item if $x$ has no such ancestor, the find method pays for that node.
\end{enumerate}

Since $\delta$ takes only values in $\{0, \ldots, \alpha(n) - 1\}$, case (2) only happens $\alpha(n)$ times per find.

We need to show that case (1) happens only $\alpha(n) \cdot {rank}(x)$ many times per node $x$.

\begin{mylemma}
For a fixed $k$, assume that case (1) has happened $i$ times for a node $x$ with $k = \delta(x)$, and let $w = {par}(x)$. Then, ${rank}(w) > A_k^{(i)}(rank(x))$.
\end{mylemma}
\begin{proof}
Induction on $i$, the number of times we charge. For $i= 0$, $A_k^{(0)} = {rank}(x) < {rank}(w)$.
Assume ${rank}(w) \ge A_k^{(i)}({rank}(x))$ and $x$ is charged again. Then, $x$ gets a new parent $z$ by path compression. Also, $\delta(x) = k = \delta(y)$ for some ancestor $y$ of $x$, with $y$ descendant of $z$.

\begin{align*}
{rank}(z) &\ge {rank}({par}(y)) \\
& \ge A_k({rank}(y)) && \text{(because $\delta(y) = k$)} \\
& \ge A_k({rank}(w)) \\
& > A_k(A_k^{(i)}({rank}(x))) && \text{(by I.H.)} \\
& = A_k^{(i+1)}({rank}(x))
\end{align*}
\end{proof}

So, after case (1) happened ${rank}(x)$ times for a fixed $k = \delta(x)$.

Then, ${rank}({par}(x)) > A_k^{({rank}(x))}({rank}(x)) = A_{k+1}({rank}(x))$. Therefore,
$\delta(x) \ge k+1$ at this point. That means, $x$ is charged only ${rank}(x)$ times for each $k\in \{0, \ldots, \alpha(n) - 1\}$, thus $\alpha(n)\cdot {rank}(x)$ charges in total.

\bigskip At the end we have the following amortized complexity:

$$O(\underbrace{n \cdot \alpha(n)}_\text{amortized cost of $n$ makesets} + \underbrace{m \cdot \alpha(n)}_\text{amortized cost of $m$ union/find}) = O((n+m) \cdot \alpha(n))$$




