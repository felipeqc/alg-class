\documentclass[a4paper]{article}
\usepackage[top=1in,bottom=1in,left=1in,right=1in]{geometry}
\usepackage{times}
\usepackage{amssymb}
\usepackage{mathtools}	% pulls in amsmath
	\mathtoolsset{centercolon}
\usepackage{tikz}
	\usetikzlibrary{automata}
\usepackage{mathpartir}
\usepackage{amsthm}
\usepackage{amsxtra}
\usepackage{algorithm}
\usepackage{algpseudocode}
\usepackage{semantic}
	\reservestyle{\declarevars}{\texttt}
	\reservestyle{\declareops}{\texttt}
	\reservestyle{\declarestates}{\text}
\usepackage{color}
\usepackage{listings}

\newtheorem{theorem}{Theorem}

\newtheorem{myexample}{\textbf{Example}}
\newtheorem{mylemma}{\textbf{Lemma}}
\newtheorem{myproof}{\textbf{Proof}}
\newtheorem{myinvariant}{\textbf{Invariant}}
\newtheorem{mytheorem}{\textbf{Theorem}}
\newtheorem{mycorollary}{\textbf{Corollary}}
\newtheorem{myapproach}{Approach}
\newtheorem{myproperty}{Property}
\newtheorem{mydefinition}{Definition}

\newtheorem{mycase}{Case}

\lstset{ %
  backgroundcolor=\color{white},   % choose the background color; you must add \usepackage{color} or \usepackage{xcolor}
  basicstyle=\footnotesize,        % the size of the fonts that are used for the code
  breakatwhitespace=false,         % sets if automatic breaks should only happen at whitespace
  breaklines=true,                 % sets automatic line breaking
  captionpos=b,                    % sets the caption-position to bottom
  commentstyle=\color{mygreen},    % comment style
  deletekeywords={...},            % if you want to delete keywords from the given language
  escapeinside={\%*}{*)},          % if you want to add LaTeX within your code
  extendedchars=true,              % lets you use non-ASCII characters; for 8-bits encodings only, does not work with UTF-8
  frame=single,                    % adds a frame around the code
  keepspaces=true,                 % keeps spaces in text, useful for keeping indentation of code (possibly needs columns=flexible)
  keywordstyle=\color{blue},       % keyword style
  language=Octave,                 % the language of the code
  morekeywords={assign, increment, decrement, jump, jump_if, store, *, +},            % if you want to add more keywords to the set
  numbers=left,                    % where to put the line-numbers; possible values are (none, left, right)
  numbersep=5pt,                   % how far the line-numbers are from the code
  rulecolor=\color{black},         % if not set, the frame-color may be changed on line-breaks within not-black text (e.g. comments (green here))
  showspaces=false,                % show spaces everywhere adding particular underscores; it overrides 'showstringspaces'
  showstringspaces=false,          % underline spaces within strings only
  showtabs=false,                  % show tabs within strings adding particular underscores
  stepnumber=1,                    % the step between two line-numbers. If it's 1, each line will be numbered
  stringstyle=\color{mymauve},     % string literal style
  tabsize=2,                       % sets default tabsize to 2 spaces
  title=\lstname                   % show the filename of files included with \lstinputlisting; also try caption instead of title
}

\newcommand{\any}{{\rule[-.2ex]{1ex}{.4pt}}}	% Less hideous than \_.
\newcommand{\RR}{\mathbb{R}}
\newcommand{\NN}{\mathbb{N}}
\newcommand{\RP}{\RR_{\ge 0}}
\newcommand*{\dave}[1]{{\color{red}\textbf{PDS: #1}}}
\newcommand{\ie}{\emph{i.e.,} }
\newcommand{\eg}{\emph{e.g.,} }
\usepackage{hyperref}
\newcommand*{\figref}[1]{\hyperref[#1]{Figure~\ref*{#1}}}

\title{Exercise Sheet 1---Algorithms and Data Structures}
\author{Felipe Cerqueira \\ 2547787 \and David Swasey \\ 2542105}

\begin{document}

\maketitle

Tutorial time: Monday 14:00

\section{Exercise 1 (10 pts)}

\noindent a) Prove the statement from the lecture: If for two functions $f, g: \NN \to \RP$, the value $L := \lim_{n \to \infty} \frac{f(n)}{g(n)} \in [0, \infty)$ exists, then

\begin{itemize}
\item if $L = 0$, then $f(n) \in o(g(n))$.
\item if $L \in (0, \infty)$, then $f(n) \in \Theta(g(n))$.
\item if $L = \infty$, then $f(n) \in \omega(g(n))$.
\end{itemize}

\bigskip \noindent \textbf{Answer:}

We use the definition of limit.
The definitions for O-notation assume that $f$ and $g$ are positive functions.
We modified the problem statement accordingly.

\begin{itemize}
\item \textbf{If $L = 0$, then $f(n) \in o(g(n))$.}

By definition of limit at infinity:

$$\forall c > 0, \exists n_0 \in \mathbb{N}, \forall n > n_0, \left | \frac{f(n)}{g(n)} - L \right | < c$$

Since $L=0$ and the functions are positive, this implies that:

$$\forall c > 0, \exists n_0 \in \mathbb{N}, \forall n > n_0, f(n) \le c \cdot g(n)$$

Thus, $f(n) = o(g(n))$.

\bigskip

\item \textbf{if $L \in (0, \infty)$, then $f(n) \in \Theta(g(n))$.}

By definition of limit at infinity:

$$\forall c > 0, \exists n_0 \in \mathbb{N}, \forall n > n_0, \left | \frac{f(n)}{g(n)} - L \right | < c$$

Thus:

$$\forall c > 0, \exists n_0 \in \mathbb{N}, \forall n > n_0,  \left | \frac{f(n) - L \cdot g(n)}{g(n)} \right | < c$$

\dave{I don't follow the reasoning.}

This leads to two possible solutions. First:

$$f(n) \le (c + L) \cdot g(n)$$

Here, $\exists c' > 0, \exists n_0 \in \mathbb{N}, \forall n \ge n_0$, $f(n) \le c' g(n)$.
That is, $f(n) = O(g(n))$.

\bigskip And as a second solution we have:

$$f(n) \ge (L - c) \cdot g(n)$$

Here $\exists c', 0 < c' < L, \exists n_0 \in \mathbb{N}, \forall n \ge n_0$, $f(n) \ge c' g(n)$. That is, $f(n) = \Omega(g(n))$.

\item \textbf{if $L = \infty$, then $f(n) \in \omega(g(n))$.}

{\color{red} TODO!!!!!!!!!!!!!!!!!!!!!}

\end{itemize}

\noindent b) Give an example of two (non-negative) functions $f,g$ with $f \in \Theta(g)$, but $\lim_{n \rightarrow \infty}\frac{f(n)}{g(n)}$ does not exist.

{\color{red} TODO!!!!!!!!!!!!!!!!!!!!!}

\bigskip \noindent \textbf{Answer:}

\noindent c) Prove the following rules for O-notation:
\begin{itemize}
\item \textbf{For any positive constant} $c$, $c \cdot f(n) = \Theta(f(n))$.

Let $c$ be any positive constant. Note that $\forall n, c f(n) \le c f(n)$. Then $\exists c' = c > 0$ and $\exists n_0 = 0$, such that $\forall n \ge n_0$, $c\cdot f(n) \le c' \cdot f(n)$. So $c f(n) = O(f(n))$.

Likewise, $\forall n, c f(n) \ge c f(n)$. Then $\exists c' = c > 0$ and $\exists n_0 = 0$, such that $\forall n \ge n_0$, $c\cdot f(n) \ge c' \cdot f(n)$. So $c f(n) = \Omega(f(n))$.



\item $\mathbf{f(n) + g(n) = \Omega(f(n))}$.

Since $g(n) \ge 0$, then $f(n) + g(n) \ge 1 \cdot f(n)$. Thus, $\exists c = 1 > 0$ and $\exists n_0 = 0$, such that $\forall n \ge n_0, g(n) + f(n) \ge c \cdot f(n)$.

\item If $\mathbf{g(n) = O(f(n))}$, then $\mathbf{f(n) + g(n) = O(f(n))}$.

Assume that $\exists c > 0$ and $\exists n_0 \in 
\mathbb{N}$, such that $\forall n \ge n_0$, $g(n) \le c \cdot f(n)$.
Then, $\forall n \ge n_0$, $f(n) + g(n) \le f(n) + c \cdot f(n) = (c+1) \cdot f(n)$.

Therefore, $\exists c' = (c+1)$ and $\exists n_0 \in 
\mathbb{N}$, such that $\forall n \ge n_0$, $f(n) + g(n) \le c' \cdot f(n)$. That is, $f(n) + g(n) = O(f(n))$.


\item $O(f(n)) \cdot O(g(n)) = O(f(n) \cdot g(n))$.

Consider any $y(n) = O(f(n))$ and any $y' = O(g(n))$.

We know that $\exists c > 0$ and $\exists n_0 \in 
\mathbb{N}$, such that $\forall n \ge n_0$, $y(n) \le c \cdot f(n)$.

Similarly, $\exists c' > 0$ and $\exists n_0' \in 
\mathbb{N}$, such that $\forall n \ge n_0'$, $y'(n) \le c' \cdot g(n)$.

Let $n_0'' = n_0 + n_0'$.

Then $\forall n \ge n_0''$,

$$y(n) \le c \cdot f(n)$$

$$y'(n) \le c' \cdot g(n)$$

That is,

$$y(n) \cdot y'(n) \le c \cdot c' \cdot f(n) \cdot g(n)$$

Therefore, $\exists c'' = c \cdot c' > 0$ and $\exists n_0'' \in 
\mathbb{N}$, such that $\forall n \ge n_0''$, $y(n)\cdot y'(n) \le c'' \cdot f(n) \cdot g(n)$. It follows that $O(f(n)) \cdot O(g(n)) = O(f(n) \cdot g(n))$.

\end{itemize}

\bigskip \noindent \textbf{Answer:}

\noindent d) Show that $a^n = o(b^n)$ and $\log_a n = \Theta (\log_b n)$ for any $1 < a < b$.

\bigskip \noindent \textbf{Answer:}

For the first case, note that $\lim_{n \rightarrow \infty}\frac{a^n}{b^n} = \lim_{n \rightarrow \infty}\left (\frac{a}{b} \right )^n = 0$, since $a/b < 1$. Thus, by the theorem from the lecture, $a^n = o(b^n)$.

For the other function:

$$\lim_{n \rightarrow \infty}\frac{\log_a n}{\log_b n} = \lim_{n \rightarrow \infty}\frac{\log_b n}{\log_b a \cdot \log_b n} = \lim_{n \rightarrow \infty} \log_a b \in (0,\infty)$$

By the theorem from the lecture, $\log_a n = \Theta(\log_b n)$.

\section{Exercise 2 (10 pts)}

\bigskip \noindent \textbf{Answer:}

\section{Exercise 3 (10 pts)}

\bigskip \noindent \textbf{Answer:}

\section{Exercise 4 (10 pts)}

\bigskip \noindent \textbf{Answer:}


\section{Exercise 5 (10 pts)}

\declareops{alloc,realloc,size,init,pushback[push\any{}back],popback[pop\any{}back],get,set,copy}%
\declarevars{var,state,buf,cnt,from,to,cursor}%
\newcommand*{\SET}{\leftarrow}%
\declarestates{Normal,Halve,Double}%
\newcommand*{\pointsto}{\hookrightarrow}%
\newcommand{\stsfig}{%
	\begin{figure}
	\begin{center}
	\begin{tikzpicture}[every node/.style=draw, node distance=3cm, thick]
		\node (N) {$\<Normal>(A, n)$};
		\node (H) [below left of=N] {$\<Halve>(A, B, c, n)$};
		\node (D) [below right of=N] {$\<Double>(A, B, c, n)$};
		\path[<->] (N) edge [bend left=45] (D);
		\path[<->] (N) edge [bend right=45] (H);
	\end{tikzpicture}
	\end{center}
	\caption{STS for algorithm 5(a).}
	\label{fig:sts}
	\end{figure}
}%
\newcommand{\codefig}{%
	\begin{figure}\centering
	\begin{minipage}{0.49\linewidth}
	\begin{algorithmic}[1]
		\Require $\<Halve>(A,B,c,n)$ and $n > 0$
		\Ensure $\<Halve>(A,B,c+1,n-1)$ or $\<Normal>(B,n-1)$
		\Procedure{pop\any{}back}{}
			\State $\<cnt> \SET \<cnt>-1$\label{Hpop}
			\State{$\<to>[\<cursor>] \SET \<from>[\<cursor>]$}
			\If{$\<cursor>+1 < \<cnt>$}
				\State{$\<cursor> \SET \<cursor>+1$}
			\Else
				\State $\<state> \SET \<Normal>$ \label{HtoN}
				\State $\<buf> \SET \<to>$
			\EndIf
		\EndProcedure
		\Statex
		\Require $\<Halve>(A,B,c,n)$
		\Ensure $\<Halve>(A,B,c,n+1)$ or $\<Normal>(A,n+1)$
		\Procedure{push\any{}back}{$v$}
			\State $\<cnt> \SET \<cnt>+1$
			\State $\<from>[\<cnt>] \SET v$
			\If{$\<cnt> = |\<to>|/2$}
				\State $\<state> \SET \<Normal>$
				\State $\<buf> \SET \<from>$
			\EndIf
		\EndProcedure
	\end{algorithmic}
	\end{minipage}
	\caption{Pseudocode for the interesting operations in state $\protect\<Halve>(A,B,c,n)$.}
	\label{fig:halveops}
	\end{figure}
}%
The problem has two parts:
\begin{itemize}
	\item[a)] Design a data structure for unbounded arrays supporting worst-case rather than amortized $\<pushback>$, $\<popback>$, $\<get>$, and $\<set>$ operations.
	
	\item[b)] Design a data structure for fixed-size arrays supporting constant-time $\<init>$, $\<get>$, and $\<set>$ operations.
	After $\<init>$, any item should have the value $\bot$.\footnote{%
		A better abstraction has $\<init>$ take an initial value; \ie $\<init>(n,v)$ creates an array of size $n$ with initial value $v$.
		This is better for \emph{clients} who no longer have to contend with $\bot$ unless they want to.
		Moreover, it admits a simpler implementation.
	}
\end{itemize}

\paragraph{Answer:}

We assume the following constant-time operations for fixed-size arrays:
\begin{itemize}
	\item $\<alloc>(n)$ allocates and returns a new array of size $n$ containing ``junk''
	\item $\<size>(A)$ returns the size of array $A$ (written $|A|$)
	\item $\<get>(A,i)$ returns the $i$th element of array $A$ when $0 \le i < |A|$ (written $A[i]$)
	\item $\<set>(A, i, v)$ sets the $i$th element of $A$ to $v$ when $0 \le i < |A|$ (written $A[i] \SET v$)
\end{itemize}
We can trivially implement the $\<size>$ operation atop more primitive arrays that lack it by storing, alongside a primitive array $\widehat A$, its size $n$.

\begin{itemize}

	\item[a)]
	One can \emph{incrementalize} amortized data structures to obtain worst-case data structures~\cite{okasaki}.
	That this is possible should come as no surprise:
	To prove amortized time bounds, we \emph{logically} schedule the work of expensive operations amongst cheaper operations.
	With additional bookkeeping, we may \emph{phsically} do so, provided the bookkeeping code does not upset our complexity goals.
	The amortized analysis shows not only that such scheduling is possible, but that the incremental work will be complete before their results are needed.
	
	In this case, the amortized algorithm stores $n$ items in a capacity $w \ge n$ fixed-sized array, doubling $w$ on $\<pushback>$ when $n=w$, and halving $w$ on $\<popback>$ when $n=w/4$.
	The doubling and halving steps allocate a new array, copy the $n$ items to it, then switch to using it.
	We seek to incrementalize these $\Theta(n)$ copy operations.
	
	\stsfig

	One way to describe our data structure is via the state-transition system in \figref{fig:sts}.
	An invariant \emph{interprets} each state by specifying what our data stucture must look like in that state.
	Our operations may rely on the invariant when they start and must guarantee it when they terminate.

	\begin{itemize}
	
	\item $\<Normal>(A, n)$:
	In this state, we proceed much as in the amortized data structure; \ie the fixed-size array $A$ stores $n$ items where $0 \le n < |A| =: w$.
	Our interpretation requires\footnote{%
		We systematically distinguish (mathematical) variables (\eg $A$, $n$) from the ``variables'' comprising our algorithm's state (\eg $\<buf>$, $\<cnt>$).
		The assertion $\<var> \pointsto v$ connects the two, saying that the program variable $\<var>$ contains the value $v$.
	}
	\begin{mathpar}
		\<state>\pointsto \<Normal> \and
		\<buf> \pointsto A \and
		\<cnt> \pointsto n \\\\
		0 \le n \le |A| \and
		\infer{i \in [0,n)}{\text{element $i$ at $A[i]$}}
	\end{mathpar}
	When inserting the $w+1$st item, we transition to $\<Double>(A,B,0,n)$ where $B$ is a fresh array allocated with $\<alloc>$ and $|B| = 2w$ and $n = w+1$.
	When removing the $w/4$th item, we transition to $\<Halve>(A,B,0,n)$, where $|B| = w/2$ and $n = w/4 - 1$.
	
	We omit the code for this state.
	
	\item $\<Halve>(A, B, c, n)$:
	In this state, we incrementally copy the relevant contents of the ``from-space'' $A$ to the ``to-space'' $B$, where the ``cursor'' $c$ says how many items we have copied and (as usual) $n$ says how many items we actually store for our client.
	We interpret this state as follows.
	\begin{mathpar}
		\<state>\pointsto \<Halve> \and
		\<from> \pointsto A \and
		\<to> \pointsto B \and
		\<cursor> \pointsto c \and
		\<cnt> \pointsto n \\\\
		0 \le c < n < |B|/2 \and
		|B| = |A|/2 \and
		\infer{i \in [0,c)}{\text{element $i$ at $A[i]$ and $B[i]$}} \and
		\infer{i \in [c, n)}{\text{element $i$ at $A[i]$}}
	\end{mathpar}
	We give pseudocode for this state's $\<pushback>$ and $\<popback>$ operations in \figref{fig:halveops}.

	The $\<popback>$ operation decrements $\<cnt>$ and performs a unit of copying work, incrementing $\<cursor>$ and transitioning to the normal state when we're done copying.
	Note that we \emph{must} transition in line~\ref{HtoN}; otherwise, we would fail to satisfy our invariant.
	
	The $\<pushback>$ operation updates $\<from>$ and increments $\<cnt>$.
	If the new value of $\<cnt>$ disagrees with our interpretation, then we \emph{abort} the on-going copy.
	This sets us up for a later transition to $\<Double>$, in case our client continues to push values.

	\item $\<Double>(A, B, c, n)$:
	This state is analogous to $\<Halve>(A, B, c, n)$, but with interpretation:
	\begin{mathpar}
		\<state>\pointsto \<Double> \and
		\<from> \pointsto A \and
		\<to> \pointsto B \and
		\<cursor> \pointsto c \and
		\<cnt> \pointsto n \\\\
		0 \le c < |A| \and
		|A| < n < |B| = 2|A| \\\\
		\infer{i \in [0,c)}{\text{element $i$ at $A[i]$ and $B[i]$}} \and
		\infer{i \in [c, |A|)}{\text{element $i$ at $A[i]$}} \and
		\infer{i \in [|A|,n)}{\text{element $i$ at $B[i]$}}
	\end{mathpar}
	We omit the corresponding code.
	\end{itemize}
	Our interpretation tells us, among other things, how to access the $i$th item in our data structure and thus how to properly implement the $\<get>$ and $\<set>$ operations for each state.
	Concretely, all our operations begin with a case analysis on the current contents of $\<state>$, branching to state-specific code.

	\codefig
	
	\item[b)]

\end{itemize}

\end{document}
