\documentclass{article}
\usepackage{times}
\usepackage{amssymb}
\usepackage{amsmath}
\usepackage{amsthm}
\usepackage{amsxtra}
\usepackage{algorithm}
\usepackage{algpseudocode}
\usepackage{color}
\usepackage{listings}

\newtheorem{theorem}{Theorem}

\newtheorem{myexample}{\textbf{Example}}
\newtheorem{mylemma}{\textbf{Lemma}}
\newtheorem{myproof}{\textbf{Proof}}
\newtheorem{myinvariant}{\textbf{Invariant}}
\newtheorem{mytheorem}{\textbf{Theorem}}
\newtheorem{mycorollary}{\textbf{Corollary}}
\newtheorem{myapproach}{Approach}
\newtheorem{myproperty}{Property}
\newtheorem{mydefinition}{Definition}

\newtheorem{mycase}{Case}

\lstset{ %
  backgroundcolor=\color{white},   % choose the background color; you must add \usepackage{color} or \usepackage{xcolor}
  basicstyle=\footnotesize,        % the size of the fonts that are used for the code
  breakatwhitespace=false,         % sets if automatic breaks should only happen at whitespace
  breaklines=true,                 % sets automatic line breaking
  captionpos=b,                    % sets the caption-position to bottom
  commentstyle=\color{mygreen},    % comment style
  deletekeywords={...},            % if you want to delete keywords from the given language
  escapeinside={\%*}{*)},          % if you want to add LaTeX within your code
  extendedchars=true,              % lets you use non-ASCII characters; for 8-bits encodings only, does not work with UTF-8
  frame=single,                    % adds a frame around the code
  keepspaces=true,                 % keeps spaces in text, useful for keeping indentation of code (possibly needs columns=flexible)
  keywordstyle=\color{blue},       % keyword style
  language=Octave,                 % the language of the code
  morekeywords={assign, increment, decrement, jump, jump_if, store, *, +},            % if you want to add more keywords to the set
  numbers=left,                    % where to put the line-numbers; possible values are (none, left, right)
  numbersep=5pt,                   % how far the line-numbers are from the code
  rulecolor=\color{black},         % if not set, the frame-color may be changed on line-breaks within not-black text (e.g. comments (green here))
  showspaces=false,                % show spaces everywhere adding particular underscores; it overrides 'showstringspaces'
  showstringspaces=false,          % underline spaces within strings only
  showtabs=false,                  % show tabs within strings adding particular underscores
  stepnumber=1,                    % the step between two line-numbers. If it's 1, each line will be numbered
  stringstyle=\color{mymauve},     % string literal style
  tabsize=2,                       % sets default tabsize to 2 spaces
  title=\lstname                   % show the filename of files included with \lstinputlisting; also try caption instead of title
}


\title{Exercise 02 - Algorithms and Data Structures}
\author{Felipe Cerqueira (2547787) \quad David Swasey}

\begin{document}

\maketitle

Tutorial time: Monday 14:00

\section{Exercise 1 (10 pts)}

\noindent a) Prove the statement from the lecture: If for two functions $f, g: \mathbb{N} \rightarrow \mathbb{R}$, the value $L := \lim_{n \rightarrow \infty} \frac{f(n)}{g(n)} \in [0, \infty)$ exists, then

\begin{itemize}
\item if $L = 0$, then $f(n) \in o(g(n))$.
\item if $L \in (0, \infty)$, then $f(n) \in \Theta(g(n))$.
\item if $L = \infty$, then $f(n) \in \omega(g(n))$.
\end{itemize}

\bigskip \noindent \textbf{Answer:}

We can prove with using the formal definition of limit. The definitions we use for O-notation assume that $f$ and $g$ are positive functions.

\begin{itemize}
\item 1) \textbf{If $L = 0$, then $f(n) \in o(g(n))$.}

By definition of limit at infinity:

$$\forall c > 0, \exists n_0 \in \mathbb{N}, \forall n > n_0, \left | \frac{f(n)}{g(n)} - L \right | < c$$

Since $L=0$ and the functions are positive, this implies that:

$$\forall c > 0, \exists n_0 \in \mathbb{N}, \forall n > n_0, f(n) \le c \cdot g(n)$$

Thus, $f(n) = o(g(n))$.

\bigskip

\item 2) if $L \in (0, \infty)$, then $f(n) \in \Theta(g(n))$.

By definition of limit at infinity:

$$\forall c > 0, \exists n_0 \in \mathbb{N}, \forall n > n_0, \left | \frac{f(n)}{g(n)} - L \right | < c$$

Thus:

$$\forall c > 0, \exists n_0 \in \mathbb{N}, \forall n > n_0,  \left | \frac{f(n) - L \cdot g(n)}{g(n)} \right | < c$$

This leads to two possible solutions. First:

$$f(n) \le (c + L) \cdot g(n)$$

Here, $\exists c' > 0, \exists n_0 \in \mathbb{N}, \forall n \ge n_0$, $f(n) \le c' g(n)$.
That is, $f(n) = O(g(n))$.

\bigskip And as a second solution we have:

$$f(n) \ge (L - c) \cdot g(n)$$

Here $\exists c', 0 < c' < L, \exists n_0 \in \mathbb{N}, \forall n \ge n_0$, $f(n) \ge c' g(n)$. That is, $f(n) = \Omega(g(n))$.

\bigskip
\item 3) if $L = \infty$, then $f(n) \in \omega(g(n))$.

{\color{red} TODO!!!!!!!!!!!!!!!!!!!!!}

\end{itemize}

\noindent b) Give an example of two (non-negative) functions $f,g$ with $f \in \Theta(g)$, but $\lim_{n \rightarrow \infty}\frac{f(n)}{g(n)}$ does not exist.

{\color{red} TODO!!!!!!!!!!!!!!!!!!!!!}

\bigskip \noindent \textbf{Answer:}

\noindent c) Prove the following rules for O-notation:
\begin{itemize}
\item \textbf{For any positive constant} $c$, $c \cdot f(n) = \Theta(f(n))$.

Let $c$ be any positive constant. Note that $\forall n, c f(n) \le c f(n)$. Then $\exists c' = c > 0$ and $\exists n_0 = 0$, such that $\forall n \ge n_0$, $c\cdot f(n) \le c' \cdot f(n)$. So $c f(n) = O(f(n))$.

Likewise, $\forall n, c f(n) \ge c f(n)$. Then $\exists c' = c > 0$ and $\exists n_0 = 0$, such that $\forall n \ge n_0$, $c\cdot f(n) \ge c' \cdot f(n)$. So $c f(n) = \Omega(f(n))$.



\item $\mathbf{f(n) + g(n) = \Omega(f(n))}$.

Since $g(n) \ge 0$, then $f(n) + g(n) \ge 1 \cdot f(n)$. Thus, $\exists c = 1 > 0$ and $\exists n_0 = 0$, such that $\forall n \ge n_0, g(n) + f(n) \ge c \cdot f(n)$.

\item If $\mathbf{g(n) = O(f(n))}$, then $\mathbf{f(n) + g(n) = O(f(n))}$.

Assume that $\exists c > 0$ and $\exists n_0 \in 
\mathbb{N}$, such that $\forall n \ge n_0$, $g(n) \le c \cdot f(n)$.
Then, $\forall n \ge n_0$, $f(n) + g(n) \le f(n) + c \cdot f(n) = (c+1) \cdot f(n)$.

Therefore, $\exists c' = (c+1)$ and $\exists n_0 \in 
\mathbb{N}$, such that $\forall n \ge n_0$, $f(n) + g(n) \le c' \cdot f(n)$. That is, $f(n) + g(n) = O(f(n))$.


\item $O(f(n)) \cdot O(g(n)) = O(f(n) \cdot g(n))$.

Consider any $y(n) = O(f(n))$ and any $y' = O(g(n))$.

We know that $\exists c > 0$ and $\exists n_0 \in 
\mathbb{N}$, such that $\forall n \ge n_0$, $y(n) \le c \cdot f(n)$.

Similarly, $\exists c' > 0$ and $\exists n_0' \in 
\mathbb{N}$, such that $\forall n \ge n_0'$, $y'(n) \le c' \cdot g(n)$.

Let $n_0'' = n_0 + n_0'$.

Then $\forall n \ge n_0''$,

$$y(n) \le c \cdot f(n)$$

$$y'(n) \le c' \cdot g(n)$$

That is,

$$y(n) \cdot y'(n) \le c \cdot c' \cdot f(n) \cdot g(n)$$

Therefore, $\exists c'' = c \cdot c' > 0$ and $\exists n_0'' \in 
\mathbb{N}$, such that $\forall n \ge n_0''$, $y(n)\cdot y'(n) \le c'' \cdot f(n) \cdot g(n)$. It follows that $O(f(n)) \cdot O(g(n)) = O(f(n) \cdot g(n))$.

\end{itemize}

\bigskip \noindent \textbf{Answer:}

\noindent d) Show that $a^n = o(b^n)$ and $\log_a n = \Theta (\log_b n)$ for any $1 < a < b$.

\bigskip \noindent \textbf{Answer:}

For the first case, note that $\lim_{n \rightarrow \infty}\frac{a^n}{b^n} = \lim_{n \rightarrow \infty}\left (\frac{a}{b} \right )^n = 0$, since $a/b < 1$. Thus, by the theorem from the lecture, $a^n = o(b^n)$.

For the other function:

$$\lim_{n \rightarrow \infty}\frac{log_a n}{log_b n} = \lim_{n \rightarrow \infty}\frac{\log_b n}{\log_b a \cdot \log_b n} = \lim_{n \rightarrow \infty} \log_a b \in (0,\infty)$$

By the theorem from the lecture, $log_a n = \Theta(\log_b n)$.

\section{Exercise 2 (10 pts)}

\bigskip \noindent \textbf{Answer:}

\section{Exercise 3 (10 pts)}

\bigskip \noindent \textbf{Answer:}

\section{Exercise 4 (10 pts)}

\bigskip \noindent \textbf{Answer:}


\section{Exercise 5 (10 pts)}

\bigskip \noindent \textbf{Answer:}

\end{document}
