\documentclass[a4paper]{article}
\usepackage[top=1in,bottom=1in,left=1in,right=1in]{geometry}
\usepackage{times}
\usepackage{amssymb}
\usepackage{mathtools}	% pulls in amsmath
	\mathtoolsset{centercolon}
\usepackage{tikz}
	\usetikzlibrary{automata}
	\usepackage{tikz-qtree}
\usepackage{mathpartir}
\usepackage{amsthm}
\usepackage{amsxtra}
\usepackage{algorithm}
\usepackage{algpseudocode}
\usepackage{semantic}
	\reservestyle{\declarevars}{\texttt}
	\reservestyle{\declareops}{\texttt}
	\reservestyle{\declarestates}{\text}
\usepackage{color}
\usepackage{listings}
\usepackage{mathtools}
\usepackage[shortlabels]{enumitem}
\usepackage{graphicx}	% for pdf image

\newtheorem{theorem}{Theorem}

\newtheorem{myexample}{\textbf{Example}}
\newtheorem{mylemma}{\textbf{Lemma}}
\newtheorem{myproof}{\textbf{Proof}}
\newtheorem{myinvariant}{\textbf{Invariant}}
\newtheorem{mytheorem}{\textbf{Theorem}}
\newtheorem{mycorollary}{\textbf{Corollary}}
\newtheorem{myapproach}{Approach}
\newtheorem{myproperty}{Property}
\newtheorem{mydefinition}{Definition}

\newtheorem{mycase}{Case}

\lstset{ %
  backgroundcolor=\color{white},   % choose the background color; you must add \usepackage{color} or \usepackage{xcolor}
  basicstyle=\small,        % the size of the fonts that are used for the code
  breakatwhitespace=false,         % sets if automatic breaks should only happen at whitespace
  breaklines=true,                 % sets automatic line breaking
  captionpos=b,                    % sets the caption-position to bottom
  commentstyle=,    % comment style
  deletekeywords={...},            % if you want to delete keywords from the given language
  escapeinside={\%*}{*)},          % if you want to add LaTeX within your code
  extendedchars=true,              % lets you use non-ASCII characters; for 8-bits encodings only, does not work with UTF-8
 % frame=single,                    % adds a frame around the code
  keepspaces=true,                 % keeps spaces in text, useful for keeping indentation of code (possibly needs columns=flexible)
  columns=fullflexible,	% not monospace
  keywordstyle=,       % keyword style
  language=Octave,                 % the language of the code
  morekeywords={forall, to, else, then, end, and, or, assign, increment, decrement, jump, jump_if, store, *, +},            % if you want to add more keywords to the set
  numbers=left,                    % where to put the line-numbers; possible values are (none, left, right)
  numbersep=5pt,                   % how far the line-numbers are from the code
  rulecolor=\color{black},         % if not set, the frame-color may be changed on line-breaks within not-black text (e.g. comments (green here))
  showspaces=false,                % show spaces everywhere adding particular underscores; it overrides 'showstringspaces'
  showstringspaces=false,          % underline spaces within strings only
  showtabs=false,                  % show tabs within strings adding particular underscores
  stepnumber=1,                    % the step between two line-numbers. If it's 1, each line will be numbered
  stringstyle=,     % string literal style
  tabsize=4,                       % sets default tabsize to 2 spaces
  title=\lstname,                  % show the filename of files included with \lstinputlisting; also try caption instead of title
  mathescape,
  belowskip=-\baselineskip,
}

\DeclareMathOperator{\prob}{prob}
\DeclareMathOperator{\dom}{dom}
\DeclareMathOperator{\rank}{rank}
\DeclareMathOperator{\key}{key}
\newcommand*{\floor}[1]{\left\lfloor{#1}\right\rfloor}
\newcommand*{\ceil}[1]{\left\lceil{#1}\right\rceil}
\newcommand{\any}{{\rule[-.2ex]{1ex}{.4pt}}}	% Less hideous than \_.
\newcommand{\RR}{\mathbb{R}}
\newcommand{\NN}{\mathbb{N}}
\newcommand{\ZZ}{\mathbb{Z}}
\newcommand{\RP}{\RR_{\ge 0}}
\newcommand*{\dave}[1]{{\color{red}\textbf{PDS: #1}}}
\newcommand{\ie}{\emph{i.e.,} }
\newcommand{\eg}{\emph{e.g.,} }
\usepackage{hyperref}
\newcommand*{\Sref}[1]{\hyperref[#1]{\S\ref*{#1}}}
\newcommand*{\figref}[1]{\hyperref[#1]{Figure~\ref*{#1}}}
\newcommand{\edge}{\longrightarrow}
\newcommand{\redge}{\longleftarrow}

\title{Exercise Sheet 9---Algorithms and Data Structures}
\author{Felipe Cerqueira \\ 2547787 \and David Swasey \\ 2542105}

\begin{document}

\maketitle

Tutorial time: Monday 14:00

\section*{Exercise 1 (10 pts)}

An edge cover of a graph $G = (V, E)$ (without isolated vertices) is a subset $F \subseteq E$ such that every vertex is incident to at least one edge in $F$. Relate edge covers of minimum cardinality to maximum matchings and provide an efficient algorithm to determine a minimum edge cover.

\paragraph{Answer.}

Any maximum matching $M$ of size $s$ is also a minimum cardinality edge cover of size $s$, because removing one more edge from $M$ would disconnect one vertex in $F$, breaking the edge cover property.

Likewise, any minimum cardinality edge cover $C$ of size $s$ is also a maximum matching of size $s$, because adding one more edge to $C$ would create two incident edges for the same vertex, breaking the matching property.

To compute a minimum cardinality edge cover, we can simply run the blossom algorithm for finding the maximum matching, since the problem is equivalent.

\section*{Exercise 2 (10 pts)}

Show that an undirected connected tree $T$ has at most one perfect matching. Show that if a tree $T$ has a perfect matching, then for every $v \in V$ , the forest $T \setminus \{v\}$ has exactly one component with an odd number of vertices.

\paragraph{Answer.}

Suppose $T$ has some perfect matching $M$. Then, there is an alternating $s$-$t$-path that goes from some node $s$ to some node $t$, passing through every node in $T$. Because in a connected tree there is exactly one single path between any pairs of nodes, there can be no other perfect matching, except for this one under the $s$-$t$ path.

If $T$ has a perfect matching, then there is an alternating path of odd length (even number of vertices) that passes through every node in $T$. Let $v$ be any vertex in this path. If we remove $v$ from the graph, there are two cases.
If $v$ was at the end-points, this leaves us with a single even-length path, which is a component with an odd number of nodes. If $v$ is an intermediate node, this leaves us with two even-length paths, i.e., two components with an odd number of nodes.

\section*{Exercise 3 (10 pts)}

Let $G = (V, E)$ be an undirected graph with vertices $V = \{v_1,\ldots , v_n\}$. Give an efficient algorithm that, given a sequence of non-negative integers $d_1, \ldots, d_n$, decides in polynomial time if $G$ admits an orientation of the edges such that the out-degree of vertex $v_i$ becomes $d_i$, for all $i = 1, \ldots, n$, and computes the orientation if it exists.

\section*{Exercise 4 (10 pts)}

A vertex cover of a graph $G=(V,E)$ is a subset $U \subseteq V$ such that for every edge $e=\{u,v\} \in E$ we have at least one of $u,v \in U$, i.e., $|U \cap e| \ge 1$. Let $U'$ be a vertex cover of G with minimum cardinality. Show how to use a maximum matching algorithm to construct a vertex cover $U$ with $|U| \le 2|U'|$.

\paragraph{Answer.}

Let $M$ be a maximum matching. Clearly, all matched vertices in $M$ are covered by the minimum number of edges (one edge for two vertices). Thus, there are $|M/2|$ vertices covered by $|M|$ edges.

In order to produce a vertex cover, let's consider any vertex $v$ not matched by $M$. It must be that every edge from $v$ goes to a matched vertex in $M$ (otherwise, $v$ could be matched and would be part of the maximum matching). In order to cover $v$, we must use one of those edges. Thus, the remaining $|V| - |M/2|$ vertices can be covered with additional $|V| - |M/2|$ edges.

Therefore, we can obtain a vertex cover with $|M/2| + |V| - |M/2| = |V|$ edges. Since in the best case each edge can cover at most two vertices, the minimum vertex cover $U'$ satisfies $|V| \le 2 |U'|$.

This implies that the constructed vertex cover has $|V| \le 2 |U'|$ edges.

\section*{Exercise 5 (10 pts)}

In a server farm you have m machines, and there are $n$ jobs that should be processed. Job $i$ has processing time $p_{ij} > 0$ if it is processed on machine $j$. The goal is to assign each job to exactly one machine and for each machine determine an ordering of jobs assigned to it. The machine then processes its assigned jobs one-by-one in the specified order. If job $i$ is put in position $k$ on machine $j$, the completion time of $i$ is the sum of processing times of jobs at positions $1, \ldots, k$ on machine $j$. Design and analyze an efficient algorithm to compute an assignment of jobs to machines and orderings such that the sum of all completion times is minimized.

\paragraph{Answer.}

%This is the same problem shown in class (similar to bin-packing/knapsack problem), except that we need to find the optimal solution. We can use dynamic programming to derive a pseudo-polynomial algorithm.

%We define a ``harder'' version of the problem, assuming each machine $j$ has a maximum capacity $k$ (instead of being unbounded).

%Let the solution $S_{i,j,k}$ denote the sum of job costs on machine $j$ assuming two conditions: (1) only jobs $1,\ldots, i$ are considered, and (2) we assume a maximum capacity $k$ for every machine.

%Given this definition, we can establish the following recurrence relation.
%If the optimal solution for $S_{i,j,k} = x$ contains job $i$, then the solution for $S_{i-1,j,k-p_{i,j}} = x - p_{i,j}$. Otherwise, $S_{i-1,j,k} = x$. And of course, we take the minimum of both solutions.

%We also account for the corner cases where the capacity

%\begin{lstlisting}[mathescape]
%\end{lstlisting}


\end{document}
