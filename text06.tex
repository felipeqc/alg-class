— Search trees

• Offer a more elegant writeup for these (2,4)-trees.

Terms:
	root, parent, internal vs leaf node, sibling node, node depth,
	ancestors/descendants,

ρ(v) := # children of v
height of tree is max depth of a leaf.

Elements have keys from a /totally ordered/ set U (eg U=ℤ).
Want to represent a sequence $S$ of /sorted/ elements taht supports
• insert(e,S)
• remove(k,S)
• search(k,S) returns min_{e∈S}{ e : key(e) ≥ k } or max_{e∈S} {e : key(e) ≤ k}

Ex: S={1,2,15,18}
	search(3,S) returns 2 or 15
	search(5,S) returns 5
	search(30,S) returns 18

Slightly more powerful than dictionaries.

Def: A tree $T$ is a (2,4)-tree if
	(i) Each leaf has the same depth (``perfectly balanced'')
	(ii) Each internal node has 2, 3, or 4 children

Note: In such a tree, every leaf has the same depth.

Example:
	5 → 2
	5 → "7|8|9"=:big
	2 → 1; 2 → 3
	big → 7; big → 8; big → 9; big → 10
With sibling links at the leaves (sorted linked list):
	1 ↔ 3 ↔ 7 ↔ 8 ↔ 9 ↔ 10

Assume Each /leaf/ stores one element.
Assume leaves are connected by a linked list. (So each leaf has next/prev sibling pointers.)

Assume internal node v stores p(v) = d stores d pointers to its children.
Moreover, it stores (d-1) keys, s₁, …, s_{d-1} called /splitters/, such that
	any key $k$ stored in the $i$th subtree of $v$ satisfies
	s_{i-1} < k ≤ s_{i}.
Convention: $s₀ = -∞$ and $s_d = +∞$.

search(k,T)
	v ← root(T)
	while v is not a leaf
		find j s.t. s_{j-1} <k ≤ s_j
		with s₁, …, s,_{ρ(v)-1} the splitters of v
		v ← jth child of v
	end while
	return v

eg, search(6,example tree) returns the leaf with key 7
and search(4,example tree) returns the leaf with key 3.

cost of search: O(depth of returned leaf) = O(height of tree)

What's the height of a (2,4)-tree?

Lemma:
	A (2,4)-tree with $n$ leaves and height $h$ satisfies
	\[
		2^h ≤ n≤ 4^h
	\]
	or, equivalently,
	\[
		\frac{1}{2} \lg n \le h \le \lgn
	\]
\begin{proof}
	By picture:
	Count the number of nodes on each level.
	Nodes of depth 0: Just the root: ≤ 1.
	Nodes of depth 1: ≤ 4¹
	Nodes of depth 2: ≤ 4²
	⋯
	Nodes of depth h: ≤ 4^h
	Thus we have at most $4^h$ leaves, but the number of leaves is $n$.
	
	For the other direction,
	depth 0:	≥ 1
	depth i:	≥ 2^i
	Number of leaves $n ≥ 2^h$.
\end{proof}

More interesting: Insert and remove.
We must maintain the invariant.

For a node n containing element e, put key(n) := key(e).

New example:
	r := "2 7 9" → 1; r → "4 5 6"; r → 8; r → 10
	1 → 1,2
	"456" → 4,5,6,7
	8 → 8,9
	10 → 10,11

When we insert 3, we want to create a new leaf.
We can use search(3) to find the right position in the tree for the new leaf.
In this case, search(3) returns the leaf 4.
We create a node 3.
Replace parent of 3 by
	x' := "34", x'' := "567".
	
insert(e,T)
	follow path from root to leaf v as in search(key(e),T)
	create a new node w for e
	put w as left- or right-sibling of v, depending on key(v) and key(w).
	Let x be the parent of v.
	In previous step, x got a new child.
	Add a splitter for $x$ at appropriate position.
		If the resulting node after insertion has at most 4 children (ρ(x)≤4), we're done; else:
		We must repair.
		Split the children into 2 and 3.
		We replace $x$ by two siblings $x'$, $x''$.
		$x'$ gets first two children.
		$x''$ gets last three children.
		Now we must assign splitters to $x'$ and $x''$.
		Make $s₁$ the splitter of $x'$.
		Make $s₃$, $s₄$ the splitters of $x''$.
		Idea: We use $s₂$ as the splitter in repairing the parent of $y$, which just got one new child.
		Let $y$ be the parent of $x$.
		Add $s₂$ as additional splitter in $y$.
		Recurse with $x := y$ and $s₂$.

There's a special case: We're at the root.
Splitting the root is strictly simpler than splitting subtrees, it involves creating a fresh root.

Cost of a single repair: O(1).
Cost of initial search: O(h).
Number of repairs: O(h).
Cost of insert: O(h) = O(log b).

remove: homework.

NB This generalizes to so-called ($A$, $B$)-trees, where every node but the root has at least $A$ and at most $B$ children.
The height of the tree is logarithmic in $B$, so we generally use much larger constants than 2 and 4.

— Splay trees

A splay tree is a binary tree (\ie $ρ(v) ≤ 2$).
We store one element per \emph{node}; \ie internal nodes also contain elements.
Invariant: At any node $v$, the keys in the left subtree $< \key(v) <$ keys in right subtree.

We do not require that the tree is balanced.

Example:
	6 → 3, 8
	3 → 1,4

Example:
	5 → 4 → 3 → 2 → 1

splay(k, T): Reorganizes $T$ so that the root of $T$ is one of
\begin{mathpar}
	min_{e∈S} {e : \key(e) ≥ k}
	max_{e∈S} {e : \key(e) ≤ k}.
\end{mathpar}
In particular, if $k$ occurs in $T$, then splay makes the node for $k$ the root of $T$.

type 'a tree ::= E | T of 'a tree * 'a * 'a tree

Assume we have splay.
We can define
	• search(k,T): splay(k,T) and return the root.
	• insert(e,T):
		Create a new node w for e.
		splay(key(e), T) and examine the root.
		Case root = E: return T(E, e, E)
		Case root = T(a, e', b):
			if key(v) ≤ key(w)
				Construct the tree T(T(a, e', E), e, b)
			else if key(v) > key(w)
				Construct the tree T(a, e, T(E, e', b))
	• remove(k, T):
		Let t := splay(k, T)
		Case root = E: E
		Case root = T(a, e', b):
			Let a' := splay(+∞, a)	; k will also do
			Idea: make b the right subtree of a'.
			case a' of
			| E => b
			| T(a', e'', b) => T(a', e'', b)

find(k, T): Binary tree search.
	x ←root
	if key(x) = k, return x.
	if key(x) < k then
		if x has no right child, return x
		else find(k, right child x)
	else
		if x has no left child, return x
		else find(k, left child x)

patch(x):
	as long as $x$ is not the root, perform one of the following rotations.
	
	(i) zig: If y, the parent of x, is the root
		rewrite
			T(T(A, x, B), y, C) → T(A, x, T(B,y,C))
		Or the symmetric case.
		observe we preserved the invariant
		
	(ii) zig-zig: If x is the left child of y, and y is the left child of z,
	(right/right) analogous
		rewrite
			T(T(T(A, x, B), y, C), z, D) →  T(A, x, T(B, y, T(C, z, D)))
	
	(iii) zig-zag: If path from z to x is "left-right" or "right-left", then rewrite
			T(T(A, y, T(B, x, C)), z, D) → T(T(A, y, B), x, T(C, z, D))

In every case, the tree satisifies the tree invariant.

spay(k,T):
	let c = find(k,T)
	patch every node on the path to the root.

Example:
		T(T(T(T(T(T(E, 1, E), 2, E), 3, E), 4, E) 5, E), 6, E)
Here are the patches that happen on splaying 1:
	zig-zig:
		T(T(T(T(T(E, 1, T(E, 2, T(E, 3, E))), 4, E) 5, E), 6, E)
We end up with
	1 → 6 right
	6 → 4 left
	4 → 2, 5
	2 → 3 right

Notation: For node $x$, let $S(x)$ be the subtree rooted at $x$.
Let $|T|$ be the number of nodes in $T$.
Define $μ(S) := \floor{\log |S|}$ and $μ(x) := μ(S(x))$.

We can interpret $μ(x)$ as the weight of $x$.
Heavier nodes are more likely to participate in rotations.

In the analysis, we will put a bank account on each /node/ rather than use a single account.

The proof will maintain the following /credit invariant/:
	Each node $x$ has a bank account with at least $μ(x)$ tokens.

Lemma:
	Let $x$ be the node that becomes root after a splay operation on tree $T$.
	Then, we need to spend at most
	\[
		3 (μ(T) - μ(x) + 1)
	\]
	tokens to pay for all rebalancing operations and for maintaining the credit invariant.

Note that $μ(x) ≥ 0$, so the bound is $≤ 3 \log n + 1$.
